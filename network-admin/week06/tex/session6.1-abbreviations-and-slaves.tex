% Beamer slide template prepared by Tom Clark <tom.clark@op.ac.nz>
% Otago Polytechnic
% Dec 2012

\documentclass[10pt]{beamer}
\usetheme{Dunedin}
\usepackage{graphicx}
\usepackage{fancyvrb}

\newcommand\codeHighlight[1]{\textcolor[rgb]{1,0,0}{\textbf{#1}}}

\title{DNS Zones 2: Abbreviations and Slaves}

\author[IN715]{Networks Administration}
\institute[Otago Polytechnic]{
  Otago Polytechnic \\
  Dunedin, New Zealand \\
}
\date{}
\begin{document}

%----------- titlepage ----------------------------------------------%
\begin{frame}[plain]
  \titlepage
\end{frame}


\section{Abbreviations}

%----------- slide --------------------------------------------------%
\begin{frame}[fragile]
  \frametitle{Abbreviations}
 There are a few ways we can abbreviate resource records.
 
 The \emph{origin} of a zone is the fully qualified domain name name of that 
 zone. 

 @ Notation: If the domain name in a record is the same as the 
 \emph{origin}, it can be abbreviated with an @.
 \begin{verbatim}
  example.com. IN SOA ns1.example.com. tech.somedomain.net. 
 
  @ IN SOA ns1.example.com. tech.somedomain.net.
    
  example.com. IN MX 10 mailserver.example.com.
   
  @ IN MX 10 mailserver.example.com.
   
 \end{verbatim}

\end{frame}

\begin{frame}
	\frametitle{Origin directive}
	
	You can explicitly define the origin of a zone by including 
	a directive as follows:
	\vspace{5mm}
	
	\texttt{\$ORIGIN example.com.}
	\vspace{5mm}
	
	This is optional since BIND can infer the value of \texttt{\$ORIGIN} 
	from \texttt{named.conf}, but it does make your zone file more self-documenting.
\end{frame}

%----------- slide --------------------------------------------------%
\begin{frame}[fragile]
  \frametitle{Appending domain names}

 You can leave the domain names off of the record names when they match the origin.

 \begin{verbatim}
   fred.example.com. IN A  123.220.44.91

   becomes

   fred IN A  123.220.44.91


   1.10.168.192.in-addr.arpa.  IN PTR pebbles.example.com.

   becomes

   1  IN PTR pebbles.example.com.
   
 \end{verbatim}
 
 Note that we don't use a trailing dot.  It is the absence of the dot that indicates that the 
 orign domain should be appended.
\end{frame}



%----------- slide --------------------------------------------------%
\begin{frame}[fragile]
  \frametitle{Repeated names}

 You can omit repeated resource names used on multiple records.
 \begin{verbatim}
   
   barney.example.com. IN A 71.44.116.17
                       IN A 123.211.16.100
 
\end{verbatim}

\end{frame}



\section{Slaves}
\begin{frame}
	\frametitle{What is a slave and why use one?}
	
	\begin{itemize}
		\item We need to run multiple DNS servers for redundancy.
		\item We could run multiple masters, but then we have to update zone information on each of them.
		\item A better practice is to run one master server that distributes zone updates to slaves.
	\end{itemize}
\end{frame}

\begin{frame}[fragile]
	\frametitle{Master configuration}
	
	\emph{file: /var/named/etc/named.conf (on master)}
	\begin{verbatim}
	"example.com" {
	    type master;
	    file "master/example.com";
	    notify yes;
	    allow-transfer { 192.168.2.100; };
	};
	\end{verbatim}
\end{frame}

\begin{frame}[fragile]
	\frametitle{Slave configuration}
	
	\emph{file: /var/named/etc/named.conf (on slave)}
	\begin{verbatim}
	"example.com" {
	    type slave;
	    file "slave/example.com";
	    masters { 172.16.5.10; };
	};
	\end{verbatim}
\end{frame}
\end{document}
