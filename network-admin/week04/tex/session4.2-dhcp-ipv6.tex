% Beamer slide template prepared by Tom Clark <tom.clark@op.ac.nz>
% Otago Polytechnic
% Dec 2012

\documentclass[10pt]{beamer}
\usetheme{Dunedin}
\usepackage{graphicx}
\usepackage{fancyvrb}
\usepackage{hyperref}

\newcommand\codeHighlight[1]{\textcolor[rgb]{1,0,0}{\textbf{#1}}}

\title{DHCP and IPv6}

\author[IN715]{Networks Administration}
\institute[Otago Polytechnic]{
  Otago Polytechnic \\
  Dunedin, New Zealand \\
}
\date{}
\begin{document}

%----------- titlepage ----------------------------------------------%
\begin{frame}[plain]
  \titlepage
\end{frame}

\begin{frame}
	\frametitle{The role of DHCP in IPv6}
	\begin{itemize}
	    \item There are a couple of options for network autoconfiguration in an IPv6 setting.
	    \item DHCP is one option, and it works in basically the same way as it does in IPv4.
	    \item Pros (relative to other IPv6 configuration options):
	    \begin{itemize}
	    	\item We can exercise explicit control over how hosts are configured.
	    	\item We can distribute more items of information than other option allow. 
	    \end{itemize}
	    \item Cons:
	    \begin{itemize}
	    	\item We must explicitly configure and manage it (But no more than we have to for IPv4).
	    \end{itemize}
	    \end{itemize}
\end{frame}



\begin{frame}[fragile]
	\frametitle{Example configuration}
	\begin{verbatim}
	default-lease-time 600;
	max-lease-time 7200; 

	subnet6 2001:db8:0:1::/64 {
	    # Range for clients
	    range6 2001:db8:0:1::129 2001:db8:0:1::254;
	    # Additional options
	    option dhcp6.name-servers fec0:0:0:1::1;
	    option dhcp6.domain-search "domain.example";
	
	    # Example for a fixed host address
	    host specialclient {
	        host-identifier option dhcp6.client-id 00:01:00:01:4a:1f:ba:e3:60:b9:1f:01:23:45;
	        fixed-address6 2001:db8:0:1::127;
	    } 
	} 
	\end{verbatim}
\end{frame}

\begin{frame}
	\frametitle{Combining IPv4 and IPv6}
	\begin{itemize}
		\item We can run two instances of \texttt{dhcpd} side-by-side on the same server.
		\item Create to configuration files, e.g. \texttt{/etc/dhcpd.conf} and \texttt{/etc/dhcpd6.conf}
		\item Start the IPv4 version normally.
		\item Start the IPv6 version with the \texttt{-6} option and direct it to use the alternate configuration file.
	\end{itemize}
\end{frame}



\begin{frame}
	Questions?
\end{frame}
\end{document}
