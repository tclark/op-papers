% Beamer slide template prepared by Tom Clark <tom.clark@op.ac.nz>
% Otago Polytechnic
% Dec 2012

\documentclass[10pt]{beamer}
\usetheme{Dunedin}
\usepackage{graphicx}
\usepackage{fancyvrb}

\newcommand\codeHighlight[1]{\textcolor[rgb]{1,0,0}{\textbf{#1}}}

\title{A Quick Overview of IPv6}

\author[IN715]{Networks Administration}
\institute[Otago Polytechnic]{
  Otago Polytechnic \\
  Dunedin, New Zealand \\
}
\date{}
\begin{document}

%----------- titlepage ----------------------------------------------%
\begin{frame}[plain]
  \titlepage
\end{frame}



%----------- slide --------------------------------------------------%
\begin{frame}
  \frametitle{OSI Model}

 \begin{itemize}
  \item Application
  \item Presentaion
  \item Session
  \item Transport
  \item \textbf{Network}
  \item Data Link
  \item Physical
 \end{itemize}

\end{frame}

%----------- slide --------------------------------------------------%
\begin{frame}
  \frametitle{IPv4}

 \begin{itemize}
  \item IPv4 was released in 1980\footnote{RFC 760} - 1981\footnote{RFC 791}.
  \item It has been tremendously successful and will continue to be used for some time.
  \item It has some problems:
    \begin{itemize}
      \item address exhaustion
      \item complicated routing
      \item poor support for security, QoS
    \end{itemize}
 \end{itemize}

\end{frame}


%----------- slide --------------------------------------------------%
\begin{frame}
  \frametitle{A new protocol was needed}

 \begin{itemize}
  \item By the early 1990s it was clear that we needed a new protocol
  \item In 1992-1993, the IETF began looking into a version of IP.
  \item Relevant RFCs began to come out in 1996.
 \end{itemize}

\end{frame}


%----------- slide --------------------------------------------------%
\begin{frame}
  \frametitle{Some features of IPv6}

 \begin{itemize}
  \item Larger address space
  \item Simplified headers
  \item Hierachical addressing and routing
  \item Improved device autoconfiguration
  \item IPSec
 \end{itemize}

\end{frame}


%----------- slide --------------------------------------------------%
\begin{frame}
  \frametitle{IPv6 addresses}

 \begin{itemize}
  \item IPv6 addressed are 128 bits long.
  \item The first 64 bits are typically used to identify the network.
  \item The second 64 bits are used for the host.
  \item Example:
        2001:0DB8:AC10:FE01:0000:0000:0000:01A6 \\
        \hspace{15mm}2001:DB8:AC10:FE01::1A6
 \end{itemize}


\end{frame}


%----------- slide --------------------------------------------------%
\begin{frame}
  \frametitle{Address types}

 \begin{itemize}
  \item Unicast
        \begin{itemize}
          \item Global
          \item Link-Local
        \end{itemize}
  \item Multicast
  \item Anycast
  \item N.B.: No broadcast 
 \end{itemize}

\end{frame}


%----------- slide --------------------------------------------------%
\begin{frame}
  \frametitle{Address autoconfiguration}

 \begin{itemize}
  \item Manual configuration and DHCP are still available for IPv6
  \item Very briefly, autoconfiguration works like this:
      \begin{enumerate}
        \item a device determines its link-local address
        \item it sends \emph{router solicitaion} messages
        \item it receives \emph{router advertisements} in response
        \item from these the device determines its \emph{network prefix}
        \item it appends its 64 bit interface id to produce its address
      \end{enumerate}
 \end{itemize}

\end{frame}



%----------- slide --------------------------------------------------%
\begin{frame}
  \frametitle{EUI-64}

 \begin{itemize}
  \item An IPv6 interface typically uses EUI-64 to obtain its interface ID.
  \item Basically, we take the 48 bit MAC address and stretch it out to 64 bits.
  \item Example:
     \begin{enumerate}
       \item Start with a MAC address:  39:A7:97:07:CB:D0
       \item Insert FFFE into the middle:  39:A7:97:FF:FE:07:CB:D0
       \item Set the seventh\footnote{The universal-local bit} to 1:  3B:A7:97:FF:FE:07:CB:D0
     \end{enumerate}
 \end{itemize}

\end{frame}



%----------- slide --------------------------------------------------%
\begin{frame}
  \frametitle{Simplified Routing}
   %\includegraphics[scale=0.5]{nw.png}
   The larger address space provided by IPv6 allows ISPs to aggregate their customers' networks with common prefixes and advertise one route.
\end{frame}

\end{document}
