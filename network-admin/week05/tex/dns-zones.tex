% Beamer slide template prepared by Tom Clark <tom.clark@op.ac.nz>
% Otago Polytechnic
% Dec 2012

\documentclass[10pt]{beamer}
\usetheme{Dunedin}
\usepackage{graphicx}
\usepackage{fancyvrb}

\newcommand\codeHighlight[1]{\textcolor[rgb]{1,0,0}{\textbf{#1}}}

\title{DNS Zones}

\author[IN715]{Networks Administration}
\institute[Otago Polytechnic]{
  Otago Polytechnic \\
  Dunedin, New Zealand \\
}
\date{}
\begin{document}

%----------- titlepage ----------------------------------------------%
\begin{frame}[plain]
  \titlepage
\end{frame}

\begin{frame}
	\frametitle{First, some review}
	
	Last time we turned on BIND and saw how it could serve as a recursive resolver. 
	We didn't have to do any additional configuration for this to work.  But how is it configured?
	
\end{frame}

\begin{frame}[fragile]
	\frametitle{The configuration}
	
	\emph{file: /var/named/etc/named.conf}
	\begin{verbatim}
	acl clients {
	    localnets;
	    ::1;
	};
	
	options {
	   listen-on    { any; };
	   listen-on-v6 { any; };
	   allow-recursion { clients; }; 
	};
	\end{verbatim}
\end{frame}
%----------- slide --------------------------------------------------%
\begin{frame}
  \frametitle{DNS Zones}

 \begin{itemize}
  \item Recall that last time we saw how the DNS hierarchy can be viewed as a tree.
  \item Each node on that tree is a \emph{DNS Zone}.
  \item A zone is composed of a set of \emph{Resource Records} of various types.
  \item Information about a particular zone is kept in a \emph{Zone File}.  These files conform to a standard format\footnote{RFC 1034 and RFC 1035}.
 \end{itemize}

\end{frame}


\begin{frame}[fragile]
	\frametitle{Zone declarations}
	In order to have BIND load a zone it must be declared.
	
	\emph{file: /var/named/etc/named.conf}
	\begin{verbatim}
	zone "foo.org.nz" {
	    type master;
	    file "master/foo.org.nz";
	}
	\end{verbatim}
	
\end{frame}
\begin{frame}[fragile]
	\frametitle{Slave zone declarations}
	Slave zones get their zone information from a master.
	
	\emph{file: /var/named/etc/named.conf}
	\begin{verbatim}
	zone "foo.org.nz" {
	    type slave;
	    masters { 10.10.1.5; };
	    file "master/foo.org.nz";
	}
	\end{verbatim}
	
\end{frame}
%----------- slide --------------------------------------------------%
\begin{frame}
  \frametitle{Zone file names}

 \begin{itemize}
  \item There is no particular requirement for how zone files are named.
  \item Best practice is to indicate the domain name for the zone in the filename, e.g., \texttt{example.com}.
 \end{itemize}

\end{frame}


%----------- slide --------------------------------------------------%
\begin{frame}
  \frametitle{Time To Live (TTL)}

 \begin{itemize}
  \item We start a zone (in BIND 9) by specifying its \emph{TTL}, e.g.,
      \begin{itemize}
        \item \texttt{\$TTL 3h}
        \item \texttt{\$TTL 1d}
        \item \texttt{\$TTL 1w}
      \end{itemize}
  \item The TTL specifies the length of time for which our zone data should be cached.
  \item A high TTL saves load on our servers, but it means that changes will take more time to propagate.
 \end{itemize}
\end{frame}



%----------- slide --------------------------------------------------%
\begin{frame}[fragile]
  \frametitle{Statement of Authority (SOA)}
 The SOA states that this server is an \emph{authoritative} source of informatiion about our zone.
 \begin{verbatim}
  example.com. IN SOA ns1.example.com. tech.somedomain.net. (
       20140821092215	; serial number
       3h               ; slave refresh 
       1h               ; slave retry
       3d               ; slave expires
       1h )             ; negative ttl
 \end{verbatim}
\end{frame}


%----------- slide --------------------------------------------------%
\begin{frame}[fragile]
  \frametitle{Nameserver (NS) records}
 NS records identify the authoritative name servers for our zone
 \begin{verbatim}
   example.com.  IN NS ns1.example.com.
   example.com.  IN NS ns2.example.com.
   example.com.  IN NS ns.otherdomain.com.
 \end{verbatim}

\end{frame}


%----------- slide --------------------------------------------------%
\begin{frame}[fragile]
  \frametitle{Address (A) records}
 A records map host names to IP addresses.
 \begin{verbatim}
   fred.example.com. IN A  123.220.44.91

   ;a host with two addresses
   barney.example.com. IN A 71.44.116.17
   barney.example.com. IN A 123.211.16.100

   ;A records can point to the same address as other A records
   ws1.example.com. IN A 71.44.116.17
   ws2.example.com. IN A 123.211.16.100
 \end{verbatim}

\end{frame}


%----------- slide --------------------------------------------------%
\begin{frame}[fragile]
  \frametitle{Alias (CNAME) records}

 A CNAME record creates an alias for another hostname
 \begin{verbatim}
   dino.example.com. IN CNAME fred.example.com. 
 \end{verbatim}

\end{frame}


%----------- slide --------------------------------------------------%
\begin{frame}[fragile]
  \frametitle{Mail Exchange (MX) records}
 MX records identify servers that receive mail for our domain
 \begin{verbatim}
   example.com. IN MX 10  wilma.example.com.  
   example.com. IN MX 20  betty.bedrock.org.
 \end{verbatim}

\end{frame}

%----------- slide --------------------------------------------------%
\begin{frame}[fragile]
	\frametitle{Text (TXT) Records}
	
	TXT records let us put public comments is a zone.
	\begin{verbatim}
	example.com. IN TXT "v=spf1 +mx a:colo.example.com/28 -all"
	\end{verbatim}
	TXT records are used, among other things, for the Sender Policy Framework (SPF) \footnote{ https://tools.ietf.org/html/rfc7208}
\end{frame}


%----------- slide --------------------------------------------------%
\begin{frame}[fragile]
  \frametitle{Reverse zone files}

 Recall that we use the in-addr.arpa domain to support reverse DNS 
 lookups.  This requires another zone file
 
 File: db.192.168.10
 \begin{verbatim}
   $TTL 3h
   10.168.192.in-addr.arpa. IN SOA ns2.example.com. tec.sdn.net (
    ... SOA stuff ... )
       10.168.192.in-addr.arpa.  IN NS ns1.example.com.
       10.168.192.in-addr.arpa.  IN NS ns2.example.com.

       1.10.168.192.in-addr.arpa.  IN PTR pebbles.example.com.
       2.10.168.192.in-addr.arpa.  IN PTR slate.example.com.
       41.10.168.192.in-addr.arpa. IN PTR bambam.rubble.com.


   
 \end{verbatim}

\end{frame}







\end{document}
