% Beamer slide template prepared by Tom Clark <tom.clark@op.ac.nz>
% Otago Polytechnic
% Dec 2012

\documentclass[10pt]{beamer}
\usetheme{CambridgeUS}
\usepackage{graphicx}
\usepackage{fancyvrb}

\newcommand\codeHighlight[1]{\textcolor[rgb]{1,0,0}{\textbf{#1}}}

\title{Introduction to DHCP}

\author[IN715]{Networks Three}
\institute[Otago Polytechnic]{
  Otago Polytechnic \\
  Dunedin, New Zealand \\
}
\date{}
\begin{document}

%----------- titlepage ----------------------------------------------%
\begin{frame}[plain]
  \titlepage
\end{frame}


%----------- frame  ----------------------------------------------%
\begin{frame}
  \frametitle{Basic Network Configuration}

  We generally configure hosts with a few standard items of network configuration.
  \begin{itemize}
    \item IP address
    \item Network mask
    \item Gateway address
    \item DNS server addresses
    \item Others:  syslog servers, proxy servers, NTP servers, etc.
  \end{itemize} 

  We don't want to do this manually if we have very many (i.e., more than one) host on our network.  We can use the Dynamic Host Configuration Protocol (DHCP) to manage this. 
\end{frame}



%----------- frame  ----------------------------------------------%
\begin{frame}
  \frametitle{DHCP Lease}
  \begin{itemize}
    \item The client outcome of the DHCP process is a \emph{Lease}.
    \item A lease is a package of configuration information that the client
          is allotted for a specified period of time.
  \end{itemize}  
\end{frame}


%----------- frame  ----------------------------------------------%
\begin{frame}
  \frametitle{DHCP Process}
  \begin{itemize}
    \item A DHCP client needing a lease begins by sending a DHCPDISCOVER to 
          the broadcast address, 255.255.255.255 with source address 0.0.0.0.
    \item DHCP servers respond to DHCPDISCOVER messages with DHCPOFFER messages. The offers contain suggested network configuration parameters.
    \item The client accepts one offer by sending a DHCPREQUEST message to 
          the offering server.
    \item The server responds with a DHCPACK containing the lease time and other parameters.  If the client request was flawed it responds with a DHCPNAK instead.
    \item The client checks that the proposed address is available by checking ARP.  If the address is already in use it complains to the server with a DHCPDECLINE message.
    \item The client may later renew or release the lease with a DHCPRENEW or DHCPRELEASE message.
  \end{itemize}  
\end{frame}

%----------- frame  ----------------------------------------------%
\begin{frame}
  \frametitle{Authoritative DHCP Servers}
  \begin{itemize}
    \item Because many devices may provide DHCP services, you can configure a
          server to identify itself as \emph{authoritative}.
    \item Clients \emph{should} prefer offers from authoritative servers.
    \item This is not a security measure, however.
  \end{itemize}  
\end{frame}


%----------- frame  ----------------------------------------------%
\begin{frame}
  \frametitle{DHCP Relay}
  \begin{itemize}
    \item Since the broadcast DHCPDISCOVER messages don't cross network
          boundaries, DHCP doesn't work across them.
    \item We get around this by running DHCP relay servers that relay
          DHCP messages to and from a central server.
  \end{itemize}  
\end{frame}


%----------- frame  ----------------------------------------------%
\begin{frame}
  \frametitle{DHCP Software}
  \begin{itemize}
    \item The ISC DHCP server package is the preferred server implementation 
          on *nix servers.
    \item Microsoft provides a server implementation for its server OSs.
  \end{itemize}  
\end{frame}

%----------- frame  ----------------------------------------------%
\begin{frame}
  \frametitle{Security Concerns}
  \begin{itemize}
    \item Rogue DHCP servers
    \item Unauthorised clients getting access to information
    \item Denial of Service by exhausting the address pool.
  \end{itemize}  
\end{frame}

\end{document}
