\documentclass{article}
\usepackage{graphicx}
\usepackage{wrapfig}

\usepackage{verbatim}
\usepackage[parfill]{parskip}
\usepackage[margin = 2.5cm]{geometry}

\usepackage[T1]{fontenc}


\begin{document}

\title{ Assignment 1 \\ DHCP \\ IN715 Networks Three}
\date{\today}
\maketitle

\section*{Introduction}
For this assignment you will build a DHCP system for a small network that will include a DHCP server on OpenBSD, a failover server on OpenBSD, and a DHCP relay server on OpenBSD.

This assignment is due on Friday, 14 August at 6:00 PM.  It is worth 10\% of your overall mark in the paper.

\section{DHCP server on OpenBSD}
You will configure the ISC DHCP server on your OpenBSD server.  Your configuration will satisfy the following requirements:

\begin{itemize}
  \item It will use foo.org.nz as the domain name;
  \item It will give two addresses for DNS servers: The address of your \texttt{bsd-server} and the address of your \texttt{bsd-relay-server};
  \item Your server will be authoritative;
  \item It will use an address pool of 172.16.5.100 - 172.16.5.150
    \begin{itemize}
      \item The gateway for this pool will be 172.16.5.2;
      \item The default lease time shall be 1 day;
      \item The maximum lease time shall be 2 days;
    \end{itemize}
  \item On the same network as the pool above, reserve an address of
        172.16.5.30 for your Linux client on that network;
  \item It will use a second  address pool of 192.168.2.50 - 192.168.2.100
    \begin{itemize}
      \item The gateway for this pool will be 192.168.2.2
      \item The default lease time shall be 8 hours
      \item The maximum lease time shall be 12 hours.
    \end{itemize}
\end{itemize}


\section{Failover setup}
Set up the DHCP server on your \texttt{router1} system as well (same requirements as above), and configure \texttt{bsd-server} and \texttt{router1} as failover peers.  Your primary server will be \texttt{bsd-server} and \texttt{[router1} will be the secondary.


\section{DHCP relay server}
A relay server is required to serve DHCP on your internal subnet.  Configure your \texttt{bsd-relay-server} on that subnet to relay DHCP between the subnet and your primary DHCP server.

\section{Assignment submission}
Create a repository on your GitHub account named ``netadmin'' (We will use it for this and for one following assignment).  Commit your \texttt{dhcpd.conf} files (You will have two) to this repository on or before the due date and time.  Tag your commit ``assignment1'' so that it will be clear which one I will assess.

After reviewing your files, the lecturer will test your DHCP implementation on your virtual machines.

\end{document}
