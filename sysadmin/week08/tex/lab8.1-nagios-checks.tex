% Beamer slide template prepared by Tom Clark <tom.clark@op.ac.nz>
% Otago Polytechnic
% Dec 2012

\documentclass[10pt]{beamer}
\usetheme{Dunedin}
\usepackage{graphicx}
\usepackage{fancyvrb}
\newcommand\codeHighlight[1]{\textcolor[rgb]{1,0,0}{\textbf{#1}}}

\title{Nagios Checks in Depth}
%\subtitle{So Your Stuff Can look Nifty}
\author[IN719]{Systems Administration}
\institute[Otago Polytechnic]{
  School of Information Technology \\
  Otago Polytechnic \\
  Dunedin, New Zealand \\
}
\date{}
\begin{document}

%----------- titlepage ----------------------------------------------%
\begin{frame}[plain]
  \titlepage
\end{frame}

%----------- slide --------------------------------------------------%
\begin{frame}
  \frametitle{Three kinds of Nagios Checks}


\begin{enumerate}
  \item Local services
  \item Network exposed services
  \item Remote services
\end{enumerate}

\end{frame}

%----------- slide --------------------------------------------------%
\begin{frame}
  \frametitle{Local Services}


\begin{itemize}
  \item Services running on the same system that runs Nagios
  \item A good way to explore the mechanics of plugins and checks
\end{itemize}

\end{frame}

%----------- slide --------------------------------------------------%
\begin{frame}[fragile]
  \frametitle{in /etc/nagios3/conf.d/localhost\_nagios2.cfg}


\begin{Verbatim}[commandchars=\\\[\]]
define service{
        use                             generic-service
        host_name                       localhost
        service_description             Disk Space
        check_command                   \codeHighlight[check_all_disks!20%!10%]
        }
\end{Verbatim}


\end{frame}

%----------- slide --------------------------------------------------%
\begin{frame}[fragile]
  \frametitle{in /etc/nagios-plugins/config/disk.cfg}

\begin{Verbatim}[commandchars=\\\[\]]
define command{
        command_name    check_all_disks
        command_line    \codeHighlight[/usr/lib/nagios/plugins/check_disk] ...
                        \codeHighlight[-w '$ARG1$' -c '$ARG2$' -e]
        }

\end{Verbatim}
\end{frame}

%----------- slide --------------------------------------------------%
\begin{frame}[fragile]
  \frametitle{check\_disk}

  This is just a simple program written in C.  We can call it manually.
\begin{Verbatim}[commandchars=\\\[\]]
root@app:~# /usr/lib/nagios/plugins/check_disk -w 20% -c 10%
DISK OK - free space: / 5887 MB (80% inode=90%);
/lib/init/rw 122 MB (100% inode=99%);
/dev 117 MB (99% inode=98%);
/dev/shm 122 MB (100% inode=99%);|
/=1405MB;6146;6914;0;7683
/lib/init/rw=0MB;97;109;0;122
/dev=0MB;93;105;0;117 /dev/shm=0MB;97;109;0;122
\end{Verbatim}
\end{frame}


%----------- slide --------------------------------------------------%
\begin{frame}
  \frametitle{Writing your own plugins}


\begin{itemize}
  \item It turns out that it's easy to write your own nagios plugins to 
	  implement checks.
  \item Write a simple program (in whatever language your choose) that 
	  conforms to a simple text-based API.
  \item Write a \texttt{.cfg} file that defines your check command and invokes
		  your program.
\end{itemize}


\end{frame}
%----------- slide --------------------------------------------------%
\begin{frame}
  \frametitle{Network Exposed Services}


\begin{itemize}
  \item Very similar to local services
  \item Nothing extra needs to be installed on the monitored systems
  \item Just connect to the service over the network and see if it works
\end{itemize}

For example, we check MySQL, but we need to dig into the check to see why
it's not working yet.

\end{frame}

%----------- slide --------------------------------------------------%
\begin{frame}[fragile]
  \frametitle{in /etc/nagios3/conf.d/ppt\_mysq\_service.cfg}

The MySQL service is not defined, so add:
\begin{Verbatim}[commandchars=\\\[\]]
#check that mysql services are running
define service {
    # stuff omitted ...
    check_command         \codeHighlight[check_mysql]
}

\end{Verbatim}
\end{frame}


%----------- slide --------------------------------------------------%
\begin{frame}[fragile]
  \frametitle{in /etc/nagios-plugins/config/mysql.cfg}

\begin{Verbatim}[commandchars=\\\[\]]
# 'check_mysql' command definition
define command{
  command_name  check_mysql
  command_line  /usr/lib/nagios/plugins/check_mysql
                -H '$HOSTADDRESS$' 
}



\end{Verbatim}
\end{frame}


%----------- slide --------------------------------------------------%
\begin{frame}
  \frametitle{Remote Services}


\begin{itemize}
  \item Sometimes we want to monitor remote services that are not exposed on a network
  \item There are a few ways to handle this, each with its pros and cons
  \item We'll consider one way, using \emph{NRPE}
  \item We will pick up this topic next time.
\end{itemize}

\end{frame}


\end{document}
