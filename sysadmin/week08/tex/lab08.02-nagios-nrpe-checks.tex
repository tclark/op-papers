\documentclass{article}   	% use "amsart" instead of "article" for AMSLaTeX format
\usepackage[margin=0.5in]{geometry}                		% See geometry.pdf to learn the layout options. There are lots.
\geometry{a4paper}                   		% ... or a4paper or a5paper or ...

\usepackage[parfill]{parskip}    		% Activate to begin paragraphs with an empty line rather than an indent
\usepackage{graphicx}				% Use pdf, png, jpg, or eps with pdflatex; use eps in DVI mode
\usepackage{enumerate}								% TeX will automatically convert eps --> pdf in pdflatex		


\title{Remote Nagios Checks With NRPE\\ IN719 Systems Administration}
\date{}							% Activate to display a given date or no date

\begin{document}
\maketitle

\section*{Introduction}
Up until now we have performed two kinds of checks with Nagios:

\begin{enumerate}
  \item We have run local checks on our \texttt{mgmt} servers for things like disk space and processor load;
  \item We have checked network-accessible services like SSH and MySQL.
\end{enumerate}

But how can we check for local properties like the latter on remote systems?  Servers aren't in the habit of reporting on free disk space over the network.  The answer is to use the \emph{Nagios Remote Plugin Executor} (NRPE) service on the remote machine we want to monitor.  Then our Nagios server can request that the remote NRPE service perform the check and report back with the result.  In this lab we will walk through the steps required to check disk space on our \texttt{db} servers.

The NRPE service could be easily abused, since basically it accepts remote requests to execute programs on the monitored server.  For this reason, we have to configure it carefully.

\section{NRPE Server configuration}
First we need to edit the file \texttt{/etc/nagios/nrpe.cfg} on our \texttt{db} server.  We'll direct the NRPE service to accept requests from our \texttt{mgmt} servers and configure the disk check.

\textbf{Step 1:} Before you start edititing your configuration file, use the \texttt{df} command to find the device id for your hard disk.  It will look something like \texttt{/div/disk/by-uuid/f8c150e9-8751-4f87-8e56-364ad5b3b8f5}.  You will need that for your configuration.

\textbf{Step 2:} Find the line that specifies \texttt{allowed\_hosts} and change it to allow requests from \texttt{mgmt.foo.org.nz} and/or \texttt{mgmt}.

\textbf{Step 3:} Find the command definition section.  It lists and configures the Nagios checks that your server will perform via NRPE.  You will see a command called \texttt{check\_hda1}.  This is nearly what we want, but we don't have an ``\texttt{hda1}".  Change the command name in square brackets to something else, like ``\texttt{check\_hd}".  Then change the command argument that specifies \texttt{/dev/hda1} to the id of your disk that you noted in step 1.

Be sure that you restart the \texttt{nagios-nrpe-server} service after altering the configuration.

\section{Nagios Server Configuration}
Now that our remote server is ready, we can configure the check in Nagios on \texttt{mgmt}.

\textbf{Step 1:} Create a new hostgroup with a name like ``Remote Disks".  Place your \texttt{db.foo.org.nz} host in this new hostgroup.

\textbf{Step 2:}  Create a new service that performs the disk check via NRPE.  The command to do this is \texttt{check\_nrpe\_1arg!check\_hd}.  This command is defined in the file \texttt{/etc/nagios-plugins/config/check\_nrpe.cfg}.  The single argument, \texttt{check\_hd}, is the name of the command we set up in the \texttt{nrpe.cfg} file on the remote system.

\section{Follow up}
Use the NRPE service to provide some more remote checks on your db server.  Then set up those checks on your other Debian servers.

NRPE service is provided on your Windows server by the NSClient++ service.  Its configuration is similar, but you'll need to take into account the different operating system when setting up your checks.

Get a comprehensive set of checks running across your network, and document how you did it so that you will be able to add more checks as we add to it.
\end{document}
