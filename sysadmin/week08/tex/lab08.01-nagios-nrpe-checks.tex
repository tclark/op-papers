\documentclass{article}   	% use "amsart" instead of "article" for AMSLaTeX format
\usepackage[margin=0.5in]{geometry}                		% See geometry.pdf to learn the layout options. There are lots.
\geometry{a4paper}                   		% ... or a4paper or a5paper or ...

\usepackage[parfill]{parskip}    		% Activate to begin paragraphs with an empty line rather than an indent
\usepackage{graphicx}				% Use pdf, png, jpg, or eps with pdflatex; use eps in DVI mode
\usepackage{enumerate}								% TeX will automatically convert eps --> pdf in pdflatex		


\title{Remote Nagios Checks With NRPE\\ IN719 Systems Administration}
\date{}							% Activate to display a given date or no date

\begin{document}
\maketitle

\section*{Introduction}
Up until now we have performed two kinds of checks with Nagios:

\begin{enumerate}
  \item We have run local checks on our \texttt{mgmt} servers for things like disk space and processor load;
  \item We have checked network-accessible services like SSH and MySQL.
\end{enumerate}

But how can we check for local properties like the latter on remote systems?  Servers aren't in the habit of reporting on free disk space over the network.  The answer is to use the \emph{Nagios Remote Plugin Executor} (NRPE) service on the remote machine we want to monitor.  Then our Nagios server can request that the remote NRPE service perform the check and report back with the result.  In this lab we will walk through the steps required to check disk space on our servers.

The NRPE service could be easily abused, since basically it accepts remote requests to execute programs on the monitored server.  For this reason, we have to configure it carefully. In this lab we will walk through setting up the NRPE service, but remember that there is \textbf{one} right way for us to do this: Prepare a Puppet module. You will want to apply the module to your app, db, and backups servers, but it makes no sense to apply it to your management server.

\section{Install the NRPE server} 
To run NRPE checks on a host, we must first install the \texttt{nagios-nrpe-server} package. This will install the \texttt{nagios-nrpe-server} service which runs under the user and group \texttt{nagios}.

\section{NRPE Server configuration}
The NRPE service is configured primarily with the file \texttt{/etc/nagios/nrpe.cfg}. There are two things we need to accomplish with this configuration.

First, we need to configure the NRPE service so that it will accept requests to perform checks from the management servers. The NRPE service will ignore requests from any host that is not explicitly allowed to request checks. Edit the \texttt{allowed\_hosts} to accept requests from your server, e.g.:

\begin{verbatim}
  allowed_hosts=group00mgmt.foo.org.nz
\end{verbatim}

Now we need to configure the NRPE server to perform the disk check. For security reasons, NRPE will only perform the checks it has been explicitly configured to do. There are some example ones in the \texttt{nagios.cfg} file, including one that is very close to what we want. Look for the line that reads:

\begin{verbatim}
  command[check_hda1]=/usr/lib/nagios/plugins/check_disk -w 20% -c 10% -p /dev/hda1
\end{verbatim}

This defines a command called \texttt{check\_hda} that checks the device \texttt{/dev/hda1} and issues a warning when only 20\% of it's space remains free and a critical alert when only 10\% remains free. The only problem is that our systems don't have a \texttt{dev/hda1} device. We can use the \texttt{df} command to see what device we do have mounted to provide the root filesystem and change that argument in this file. We probably want to change the command's name as well.

Once these changes are done, it is necessary to restart \texttt{nagios-nrpe-server} so that they will take effect.



\section{Nagios Server Configuration}
Now that our remote server is ready, we can configure the check in Nagios on the management server.

First, create a new hostgroup with a name like ``Remote Disks".  Place your \texttt{group00db.foo.org.nz} host in this new hostgroup.

Next, reate a new service that performs the disk check via NRPE.  The command to do this is \texttt{check\_nrpe\_1arg!check\_hd}.  This command is defined in the file \texttt{/etc/nagios-plugins/config/check\_nrpe.cfg}.  The single argument, \texttt{check\_hd}, is the name of the command we set up in the \texttt{nrpe.cfg} file on the remote system. You may have named yours something different.

\section{Follow up}
Once you have NRPE and Nagios reporting on disk space on your db server, apply the NRPE module to your app and backups server and add them to the associated hostgroup. Then add other NRPE checks until you are satisfied that you are monitoring the states of all your servers thoroughly.

\end{document}
