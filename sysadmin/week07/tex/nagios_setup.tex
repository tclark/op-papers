The nagios 3 package should be installed on your mgmt server.

Set up web auth:

sudo htpasswd -c /etc/nagios3/htpasswd.users nagiosadmin

Now log in.  Loook in particular at Hosts and Hostgroups.  Nagios is configured
to monitor the local machine (mgmt) and nothing else.

We want to be able to monitor the db server.  We will modify and add files to
our nagios-server puppet modules config class

1.  Create a contact for yourself in contacts_nagios2.cfg

nagios_contact { 'tclark':
              target => '/etc/nagios3/conf.d/ppt_contacts.cfg',
              alias => 'Tom Clark',
              service_notification_period => '24x7',
              host_notification_period => '24x7',
              service_notification_options => 'w,u,c,r',
              host_notification_options => 'd,r',
              service_notification_commands => 'notify-service-by-email',
              host_notification_commands => 'notify-host-by-email',
              email => 'root@localhost',
  }


We will change that email address in a later lab.


 nagios_contactgroup { 'sysadmins':
               target => '/etc/nagios3/conf.d/ppt_contactgroups.cfg',
               alias => 'Systems Administrators',
               members => 'tclark', 
  }



Now we will define a nagios host for our db server

 nagios_host { 'db.sqrawler.com':
                 target => '/etc/nagios3/conf.d/ppt_hosts.cfg',
                 alias => 'db',
                 address => '10.25.1.18',
                 check_period => '24x7',
                 max_check_attempts => 3,
                 check_command => 'check-host-alive',
                 notification_interval => 30,
                 notification_period => '24x7',
                 notification_options => 'd,u,r',
                 contact_groups => 'sysadmins',
              }

You'll have to change the IP address 

Now define your service:

nagios_service {'MySQL':
              service_description => 'MySQL DB',
              hostgroup_name => 'db-servers',
              target => '/etc/nagios3/conf.d/ppt_mysql_service.cfg',
              check_command => 'check_mysql',
              max_check_attempts => 3,
              retry_check_interval => 1,
              normal_check_interval => 5,
              check_period => '24x7',
              notification_interval => 30,
              notification_period => '24x7',
              notification_options => 'w,u,c',
              contact_groups => 'sysadmins',
  }


And finally, define a host group

nagios_hostgroup{'db-servers':
              target => '/etc/nagios3/conf.d/ppt_hostgroups.cfg',
              alias => 'Database Servers',
              members => 'db.sqrawler.com',
  }



Other steps:

sudo chown root:puppet /etc/nagios3/conf.d
sudo chmod 775 /etc/nagios3/conf.d 

Have puppet config your server and verify that the right files are in conf.d

sudo chmod 644 /etc/nagios3/conf.d/* 

If you have done everything right, your db server will appear to be up in Nagios,
but the MySQL service will appear to be down.


