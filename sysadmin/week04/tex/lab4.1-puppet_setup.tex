\documentclass{article}
\usepackage{graphicx}
\usepackage{wrapfig}
\usepackage{inconsolata}
\usepackage{enumerate}
\usepackage{verbatim}
\usepackage[parfill]{parskip}
\usepackage[margin = 2.5cm]{geometry}

\usepackage[T1]{fontenc}


\begin{document}

\title{Lab 4.1: Puppet - Initial Setup \\ IN719 Systems Administration}
\date{}
\maketitle

\section*{Introduction}
Puppet is a tool that allows us to centralise and automate the configuration of computers.  Set up properly, it will save you time and help you keep your server configurations consistent and accurate.  Puppet is an extremely powerful (from the Greek \emph{powerei}, meaning ``This is going to be a bitch to get working."

Puppet uses a central server, called the \emph{puppetmaster}, to manage clients, called \emph{agents}.

\section{Set up the puppetmaster}
On your management server, use \texttt{apt-get} to install the \texttt{puppetmaster} package and its dependencies.  Edit the file \texttt{/etc/puppet.conf} and add the following lines:

\
\begin{verbatim}
  [master]
  certname=mgmt.sqrawler.com
\end{verbatim}

Next, create the file \texttt{/etc/puppet/manifests/site.pp}.  It should be empty for now.

Restart puppetmaster with the command \texttt{sudo /etc/init.d/puppetmaster restart}.

\section{Installing and connecting an agent}
On your db server, use \texttt{apt-get} to install the \texttt{puppet} package and its dependencies.  Connect your agent to the puppet server manually with the following command:

\begin{verbatim}
puppet agent --server=mgmt.sqrawler.com --no-daemonize --verbose
\end{verbatim}

You should see output like the following:

\begin{verbatim}
info: Creating a new SSL key for db.sqrawler.com
info: Caching certificate for ca
info: Creating a new SSL certificate request for db.sqrawler.com
info: Certificate Request fingerprint (md5): A1:2C:E1:C1:2F:C5:AE:34:A9:A5:4F:C9:CA:7A:16:C6

\end{verbatim}

Now, on the puppetmaster, run the command \texttt{puppet cert --list}.  You should see the signing request for your agent.  Sign the key with the command \texttt{ puppet cert --sign db.sqrawler.com}.

\section{Sample agent configuration}
Now we want to get Puppet to do something on our systems.  To begin, make sure that \texttt{vim} is not installed on your db server by running \texttt{sudo apt-get remove vim}.

Next create the file \texttt{/etc/puppet/mamifests/nodes.pp} on your management server and put the following text in it:
\begin{verbatim}
  node 'db.sqrawler.com' {
    package { 'vim': ensure => present }
  }
\end{verbatim}

This directs Puppet to be sure that \texttt{vim} is installed on our agent - Or at least it will, when you add the following line to \texttt{/etc/puppet/manifests/site.pp}:

\begin{verbatim}
  import 'nodes.pp'
\end{verbatim}

It is not necessary to restart the puppetmaster, but if you restart the agent (ctrl-c, up-arrow, enter) it will speed the process along.  Log into your db server in another session and check to see that \texttt{vim} was installed.






\end{document}
