\documentclass{article}   	% use "amsart" instead of "article" for AMSLaTeX format
\usepackage[margin=0.5in]{geometry}                		% See geometry.pdf to learn the layout options. There are lots.
\geometry{a4paper}                   		% ... or a4paper or a5paper or ...

\usepackage[parfill]{parskip}    		% Activate to begin paragraphs with an empty line rather than an indent
\usepackage{graphicx}				% Use pdf, png, jpg, or eps with pdflatex; use eps in DVI mode
\usepackage{enumerate}								% TeX will automatically convert eps --> pdf in pdflatex		


\title{Lab 04.2: A Puppet Module to Manage MariaDB \\ IN719 Systems Administration}
\date{}							% Activate to display a given date or no date

\begin{document}
\maketitle

\section*{Introduction}
In this lab we will build a more complex module to manage our database server software.  This module will use a collection of related \emph{classes}.  We've used classes already, but you may not have paid much attention to them.  Note that this lab is based on an example from \emph{Pro Puppet} by James Turnbull.  In this module we will handle not just installation and configuration of a service, but also preinstallation tasks, operation, and ongoing maintenance of the service.
\section{Module setup}
Create a standard module structure with the following files and directories in the \texttt{/etc/puppet/code/modules} directory of your puppetmaster.

\texttt{mariadb} \\
\texttt{mariadb/files/50-server.cnf} \\
\texttt{mariadb/manifests/init.pp} \\
\texttt{mariadb/manifests/install.pp} \\
\texttt{mariadb/manifests/config.pp} \\
\texttt{mariadb/manifests/service.pp} \\
\texttt{mariadb/templates} \\

Notice that we're using some more manifest files than we've used in the past. We will be writing a bit more code and we need to organise it more deliberately. Also, note that you can get a copy of the \texttt{50-server.cnf} file from the \texttt{week04} subdirectory of the class GitHub repository.

\section{mariadb::install}
The \textbf{mariadb::install} class includes the resources needed to install MariaDB.  Put the following in your \texttt{install.pp} file

\begin{verbatim}
  class mariadb::install {
    package { "mariadb-server" :
              ensure => present,
              require => User["mysql"],
    }
    user { "mysql":
           ensure => present,
           comment => "MariaDB user",
           gid => "mysql",
           shell => "/bin/false",
           require => Group["mysql"],
    }
    group { "mysql" :
            ensure => present,
    }
   }

\end{verbatim}

It's not a typo that the user and group are "\texttt{mysql}".

Note how we use \texttt{require} directives to make sure that things are set up in the correct order, and we don't bother attempting steps that will fail because prerequisites are not met.



\newpage

\section{mariadb::config}
Place the following resources in your \texttt{config.pp} file.

\begin{verbatim}
  class mariadb::config {
    file { "/etc/mysql/mariadb.conf.d/50-server.cnf":
      ensure => present,
      source => "puppet:///modules/mariadb/50-server.cnf",
      mode => "0444",
      owner => "root",
      group => "root",
      require => Class["mariadb::install"],
      notify => Class["mariadb::service"],
    }
  }
\end{verbatim}

Notice how these resources require mariadb::install, and they also \emph{notify} mariadb::service.  The \texttt{require} directive means that Puppet won't apply the \texttt{config} class if the \texttt{install} class hasn't been applied successfully. The \texttt{notify}directive means that the server daemon will be restarted whenever Puppet changes its configuration.

\section{mariadb::service}
The maria::service class is brief.  Place it in your \texttt{service.pp} file.

\begin{verbatim}
  class mariadb::service {
    service { "mysql" :
      ensure => running,
      hasstatus => true,
      hasrestart => true,
      enable => true,
      require => Class["mariadb::config"],
    }
  }
\end{verbatim}

This class will make sure that the server daemon is running and will restart it if necessary when its configuration is changed by Puppet.

\section{mariadb class}

Finally we just combine our classes in the \texttt{init.pp} file.

\begin{verbatim}
  class mariadb {
    include mariadb::install, mariadb::config, mariadb::service
  }
\end{verbatim}

Now you can apply the module to your db server by placing \texttt{include mariadb} in the node definition for your db server. Don't include this module in other nodes because we don't want to install or run the MariaDB server on them.

\end{document}
