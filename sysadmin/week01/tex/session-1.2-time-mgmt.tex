% Beamer slide template prepared by Tom Clark <tom.clark@op.ac.nz>
% Otago Polytechnic
% Dec 2012

\documentclass[10pt]{beamer}
\usetheme{Dunedin}
\usepackage{graphicx}
\usepackage{fancyvrb}

\newcommand\codeHighlight[1]{\textcolor[rgb]{1,0,0}{\textbf{#1}}}

\title{Time Management}

\author[IN719]{Systems Administration}
\institute[Otago Polytechnic]{
  Otago Polytechnic \\
  Dunedin, New Zealand \\
}
\date{}
\begin{document}

%----------- titlepage ----------------------------------------------%
\begin{frame}[plain]
  \titlepage
\end{frame}

%----------- slide --------------------------------------------------%
\begin{frame}
  \frametitle{}

   Why are we talking about this?
      
   \begin{itemize}
     \item So we can be professionally successful.
     \item So we can be happy.
   \end{itemize}
   
   Also, you're going to be assessed on this.
   
\end{frame}



%----------- slide --------------------------------------------------%
\begin{frame}
  \frametitle{}

   Managing one's time as a systems administrator is difficult for the same
   reason that it is interesting. Sysadmin work is 
   
   \begin{itemize}
     \item chaotic,
     \item subject to interruption,
     \item broad in scope.
   \end{itemize}
\end{frame}

%----------- slide --------------------------------------------------%
\begin{frame}
  \frametitle{}

  A good sysadmin is expected to 
  
  \begin{itemize}
    \item be available and responsive
    \item get things done
  \end{itemize}
  
  But these two things generally conflict

\end{frame}

%----------- slide --------------------------------------------------%
\begin{frame}
  \frametitle{Use a diary}

  Your diary can be a physical notebook, a calendar app, or sommething
  else provided you use it. 
  
  \begin{itemize}
    \item Track \textbf{all} your tasks, for work, school, and personal things.
    \item It should have a calendar-like property.
    \item Don't use a todo app that deletes things that are done.
  \end{itemize}
    
\end{frame}

%----------- slide --------------------------------------------------%
\begin{frame}
  \frametitle{Use your diary}
 
   \begin{itemize}
    \item Start your day by planning your tasks
    \item End your day by reviewing what you did - and did not - get done.
  \end{itemize}
        
\end{frame}

%----------- slide --------------------------------------------------%
\begin{frame}
  \frametitle{Use your diary}
 
   \begin{itemize}
    \item Start your day by planning your tasks
    \item End your day by reviewing what you did - and did not - get done.
  \end{itemize}
        
\end{frame}

%----------- slide --------------------------------------------------%
\begin{frame}
  \frametitle{Key assumption number one}

  Nobody really knows what sysadmins do, and for the most part they're happier 
  not knowing. But they do need to be confident that we know what we're doing and
  that we're doing it well. Good time management will help give them that confidence, 
  which in turn give you more freedom to manage your work.
    
\end{frame}

%----------- slide --------------------------------------------------%
\begin{frame}
  \frametitle{Day planning tips}

  \begin{itemize}
    \item If it had to be done today, do it first.
    \item Look for a balance between small easy tasks and big hard ones.
    \item Schedule in uninterruptible and interruptible times.
  \end{itemize}    
\end{frame}

%----------- slide --------------------------------------------------%
\begin{frame}
  \frametitle{First hour rule}

  The first hour of your work day is typically your most productive. Use that time 
  for important tasks and things that require focus. Do not use it to check your email,
  fetch coffee, or flirt with the new person in HR.
      
\end{frame}

%----------- slide --------------------------------------------------%
\begin{frame}
  \frametitle{Mornings are special}
  
  The first hour or two of your workday are likely to be the most productive.
  Generally that's because nothing has gone wrong yet and nobody has come up
  with a reason to interrupt you. Use your mornings well.
  
  \begin{itemize}
    \item Avoid meetings in the first two hours of the day.
    \item Do not start the day by reading email. Skim your inbox for
    anything critical and then close it.
    \item Take the fist ten minutes of your day to make a plan.
  \end{itemize}
     
\end{frame}


%----------- slide --------------------------------------------------%
\begin{frame}
  \frametitle{Use blockers}
  
  You are going to be interrupted to handle issues. It's literally part of
  your job description. The purpose of blockers is to route those issues into
  a lane where you can deal with them effectively.
  
  Examples of blockers:
  
  \begin{itemize}
    \item A ticketing system
    \item A team member taking a turn in the triage role.
    \item Policies about when you can and can't be interrupted.
  \end{itemize}
     
\end{frame}

%----------- slide --------------------------------------------------%
\begin{frame}
  \frametitle{Handling the interruptions}
  
    When something does come up, you need a decisive and efficient 
    way to respond.
    
    \begin{itemize}
    \item Record it
    \item Delegate it
    \item Do it    
    \end{itemize}
     
\end{frame}

%----------- slide --------------------------------------------------%
\begin{frame}
  \frametitle{Handling big tasks}
  
   Ideally, you start working on a task and continue until it is done.
   The world is not ideal, especially for sysadmins. However,
   
   \begin{itemize}
     \item Don't start working on a task without a plan and a schedule
     to see it through to completion.
     \item Try to schedule some time to work on the task almost every day.
     \item For especially big or complex tasks, take a team member out of
     on call rotation so that they can focus on the big task.
   \end{itemize}
   
\end{frame}
%----------- slide --------------------------------------------------%
\begin{frame}
  \frametitle{Build routines}
  
    If there are things you do on a regular basis, develop a routine of doing 
    those things at the same time every day/week/whatever.
    
    \begin{itemize}
      \item It saves mental energy.
      \item Stuff gets done.
    \end{itemize}
     
\end{frame}

%----------- slide --------------------------------------------------%
\begin{frame}
  \frametitle{Build habits}
  
    The value of (good) habits is that they lead us to do the right thing without 
    taking the time to think about it. What things should become habit?   
    
    If you find yourself, and especially if you find your self often asking ``Should I ...?''
    \begin{itemize}
      \item The answer is yes.
      \item Quit wasting time asking and make it a habit.
    \end{itemize}
     
\end{frame}

%----------- slide --------------------------------------------------%
\begin{frame}
  \frametitle{Create policies}
  
    Policies sound annoying a bureaucratic, but they save time and energy.
    
    Critical policies are 
    
    \begin{itemize}
      \item How do users get help?
      \item What is and is not supported?
      \item What is an emergency?
    \end{itemize}
     
\end{frame}






\end{document}
