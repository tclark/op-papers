\documentclass{article}
\usepackage{graphicx}
\usepackage{wrapfig}
\usepackage{inconsolata}
\usepackage{enumerate}
\usepackage{hyperref}
\usepackage{verbatim}
\usepackage[parfill]{parskip}
\usepackage[margin = 2.5cm]{geometry}

\usepackage[T1]{fontenc}


\begin{document}

\title{Lab 02.01\\Debian Packages\\IN719 Systems Administration}
\date{}
\maketitle

\section{Introduction}
One of the nice features of Debian is its excellent package management system
that makes it easy to install, update, and remove software packages. Most of 
the time we can get the software we want from official repositories. But 
sometimes we need to go beyond the standard sources and install something 
special. In these cases, we might go outside the package system, but then we 
lose the benefits of the package manager.  It turns out that it's not hard to 
create our own packages, and we'll see that in this lab.  

\section{The problem}
In the \texttt{week02} folder on GitHub there's a simple shell script called \texttt{interactive\_cowsay}. We want to
install it in \texttt{/usr/bin}. But our script has a \emph{dependency}, the Debian package \texttt{cowsay} that is available from
the standard repositories. We'll create a \texttt{.deb} fille for our packages that addresses these requirements. To
begin, copy the \texttt{interactive\_cowsay} script onto the lab server. Create a subdirectory of your home
directory called \texttt{interactive\_cowsay\_package}. We'll use it to set up our package.

\section{Preparing the package}
\begin{enumerate}
	\item We want to install the executable file under \texttt{/usr/bin}, so create a \texttt{usr/bin} subdirectory tree under
		\texttt{interactive\_cowsay\_package}. Move the \texttt{interactive\_cowsay} file into \texttt{interactive\_cowsay\_package/usr/bin}.
	\item It's best to get the ownership and permissions right at this stage. Make sure that \texttt{root} owns the \texttt{usr} sub-
		directory and everything under it. Make sure that everybody can read and execute \texttt{interactive\_cowsay}.
	\item Create another subdirectory of \texttt{interactive\_cowsay\_package} called \texttt{DEBIAN}.
	\item Inside \texttt{DEBIAN}, create three text files: \texttt{control}, \texttt{postinst}, and \texttt{prerm}.
		\texttt{postinst} is a shell script that does whatever tasks must be done after the package files are installed.
		\texttt{prerm} is a shell script that does whatever must be done before removing package files when uninstalling.
		We don't need to do anything in those cases, so both files just need to contain the following:

		\begin{verbatim}
		#!/bin/sh
		exit 0
		\end{verbatim}
\newpage

	\item The \texttt{control} file contains information about your package. Use the following information, it is explained below.
		\begin{verbatim}
		Package: interactive-cowsay
		Version: 1.0
		Architecture: all
		Essential: no
		Depends: cowsay
		Installed-Size: 512
		Maintainer: Tom Clark <tom.clark@op.ac.nz> (use your own information here)
		Description: Provides an interactive front end for cowsay
		\end{verbatim}

		\begin{description}
			\item[Package]What you chose to name your package. It shouldn't conflict with another package's name and it can only contain alphanumeric characters, hyphens, and full stops.
                        \item[Version]Whatever numbering scheme you want, but the standard x.y numeric format is best. Don't use hyphens, e.g., 2.4-2.

			\item[Architecture]Our package works on all hardware architectures.

			\item[Essential]Our package is not essential. If you try to uninstall essential packages, you will get a warning message.

			\item[Depends]Our package depends on the package \texttt{cowsay} to work properly.
				
			\item[Installed-Size]Our package will take up 512 bytes of disk once it's installed.

			\item[Maintainer]Whom to blame for this fiasco.

			\item[Description]A short description of the package.
		\end{description}
	\item Now you're ready to make the package. Move back up to your home directory, or wherever the directory \texttt{interactive\_cowsay\_package} is located. Issue the command \texttt{dpkg -b interactive\_cowsay\_package interactive-cowsay-1.0.deb}. This should produce a \texttt{.deb} file named \texttt{interactive-cowsay-1.0.deb}.
	\end{enumerate}
	\section{Installing the package}
	Now you can install the package with the command \texttt{sudo dpkg -i interactive-cowsay-1.0.deb}. Try it.
		What happens?
		The problem is that your package depends on the \texttt{cowsay} package. But dpkg isn't smart enough to find and install it for you. For that, you need \texttt{apt-get}, and to use it for your new package, you need to set up a Debian \emph{repository}. Setting up and running a repository is not hard, but it's beyond the scope of today's exercise.  However, you can complete the install by installing \texttt{cowsay} yourself and then installing your new package.




\end{document}
