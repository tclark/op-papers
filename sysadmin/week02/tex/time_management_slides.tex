% Beamer slide template prepared by Tom Clark <tom.clark@op.ac.nz>
% Otago Polytechnic
% Dec 2012

\documentclass[10pt]{beamer}
\usetheme{Dunedin}
\usepackage{graphicx}
\usepackage{fancyvrb}

\newcommand\codeHighlight[1]{\textcolor[rgb]{1,0,0}{\textbf{#1}}}

\title{Time Management}

\author[IN719]{Systems Administration}
\institute[Otago Polytechnic]{
  Otago Polytechnic \\
  Dunedin, New Zealand \\
}
\date{}
\begin{document}

%----------- titlepage ----------------------------------------------%
\begin{frame}[plain]
  \titlepage
\end{frame}

%----------- slide --------------------------------------------------%
\begin{frame}
  \frametitle{}

   Managing one's time as a systems administrator is difficult for the same
   reason that it is interesting. Sysadmins' work is unpredictable and 
   driven by interruptions.
\end{frame}

%----------- slide --------------------------------------------------%
\begin{frame}
  \frametitle{Three task types}

  From a time management perspective, we can group tasks in three groups.
  
  \begin{itemize}
  \item Routine tasks
  \item Planned projects
  \item Responding to unpredictable events
  \end{itemize}
  
  All three of these things are happening all the time.

\end{frame}

%----------- slide --------------------------------------------------%
\begin{frame}
  \frametitle{Time management approaches}

  We're going to use three big ideas to help with time management.
  
  \begin{itemize}
    \item Tools (mostly software)
    \item Work practices
    \item Implicit strategies and habits
  \end{itemize}
  
  But we're not so much going to split them up for discussion. Instead
  we'll try to see how these thing weave together.
  
\end{frame}


%----------- slide --------------------------------------------------%
\begin{frame}
  \frametitle{Key assumption number one}

  Nobody really knows what sysadmins do, and for the most part they're happier 
  not knowing. But they do need to be confident that we know what we're doing and
  that we're doing it well. Good time management will help give them that confidence, 
  which in turn give you more freedom to manage your work.
    
\end{frame}

%----------- slide --------------------------------------------------%
\begin{frame}
  \frametitle{Key assumption number two}

  Nobody cares how much you work or how hard your work. They only 
  care about \emph{what gets done}.
    
\end{frame}

%----------- slide --------------------------------------------------%
\begin{frame}
  \frametitle{Key assumption number three}

  People expect easy things to be done quickly and hard things to take
  longer. They don't always know which things are easy and which things are hard.
  To some extent we can teach this.
      
\end{frame}

%----------- slide --------------------------------------------------%
\begin{frame}
  \frametitle{Mornings are special}
  
  The first hour or two of your workday are likely to be the most productive.
  Generally that's because nothing has gone wrong yet and nobody has come up
  with a reason to interrupt you. Use your mornings well.
  
  \begin{itemize}
    \item Avoid meetings in the first two hours of the day.
    \item Do not start the day by reading email. Skim your inbox for
    anything critical and then close it.
    \item Take the fist ten minutes of your day to make a plan.
  \end{itemize}
     
\end{frame}

%----------- slide --------------------------------------------------%
\begin{frame}
  \frametitle{How to plan your day}
  
  First, you need some sort of diary or day planner. This can be a paper one,
  an online planner, or a mobile app. Just find one that works and use it.
  
  \begin{itemize}
    \item Maintain a global todo list of outstanding tasks with estimates of time required for each one. 
    Note which of the three types (routine, project, issues) of tasks each one is.
    \item There are a few tasks that need to be done today. Do those first.
    \item Look for some low hanging fruit - things that can be done in 30 minutes or less - and
    put a couple into your plan for the day. Pay special attention to items that are
    visible to your users. 
    \item Leave some blocks of time open to respond to issues that arise. Pick a low priority or
    annoying task to work on during such times if there are no issues\footnote{It could happen}.
  \end{itemize}
  
  At the end of your day, review what you got done. Anything you planned to complete and
  didn't should be a high priority tomorrow.
     
\end{frame}

%----------- slide --------------------------------------------------%
\begin{frame}
  \frametitle{Use blockers}
  
  You are going to be interrupted to handle issues. It's literally part of
  your job description. The purpose of blockers is to route those issues into
  a lane where you can deal with them effectively.
  
  Examples of blockers:
  
  \begin{itemize}
    \item A ticketing system
    \item A team member taking a turn in the triage role.
    \item A daily schedule that includes uninterruptible \emph{and} 
    interruptible blocks of time.
  \end{itemize}
     
\end{frame}

%----------- slide --------------------------------------------------%
\begin{frame}
  \frametitle{Automation}
  
    We know we're supposed to automate things but how do we go about it?
    
    \begin{itemize}
    \item Use your ticketing system and schedule planner to identify things
    you do often. Those are candidates for automation.
    \item Document the procedures for those things. Documentation is just a low
    budget form of automation.
    \item Refer to the documentation when you execute the procedures. Use
    this to refine the documentation.
    \item When your documentation is well proven, you have a good spec for 
    an automated process.
    \end{itemize}
     
\end{frame}

%----------- slide --------------------------------------------------%
\begin{frame}
  \frametitle{Efficient email}
  
  Email can suck up a huge amount of your productivity. But we've been
  using email for over 50 years, so the tools and methods for handling it are 
  pretty well developed.
  
  \begin{itemize}
    \item Use rules to pre-sort your email.
    \item Most email is, by definition, not urgent. Check your email
    a few times a day when it won't disrupt other work.
    \item Don't bother looking at you email when you don't have the time 
    to fully handle it.
    \item Set up rules to archive your email so it doesn't sit in your inbox forever.
  \end{itemize}
  
         
\end{frame}



%----------- slide --------------------------------------------------%
\begin{frame}
  \frametitle{Build habits}
  
    The value of (good) habits is that they lead us to do the right thing without 
    taking the time to think about it. Here's an example: When I open a new terminal 
    window, I \emph{always} start tmux immediately.
    
    \begin{itemize}
    \item It takes less time to type ``\texttt{tmux}'' than is does to decide 
    I didn't need it after all.
    \item When I think, ``I'm just going to do one thing; there's no point
    in starting tmux,'' I'm wrong a good fraction of the time. Then I have to 
    adjust what I'm doing and start tmux.
    \item It's a small thing, but the accumulation of small savings in time and mental 
    energy adds up.
    
    \end{itemize}
     
\end{frame}

%----------- slide --------------------------------------------------%
\begin{frame}
  \frametitle{Only handle a task once}
  
   Ideally, you start working on a task and continue until it is done.
   The world is not ideal, especially for sysadmins. However,
   
   \begin{itemize}
     \item Don't start working on a task without a plan and a schedule
     to see it through to completion.
     \item Try to schedule some time to work on the task almost every day.
     \item For especially big or complex tasks, take a team member out of
     on call rotation so that they can focus on the big task.
   \end{itemize}
   
\end{frame}


%----------- slide --------------------------------------------------%
\begin{frame}
  \frametitle{Do not be distracted by shiny things}
  
   As tech people we like the novelty that comes from building and using 
   new things. But sometimes that's a losing proposition.
   
   \begin{itemize}
     \item There's efficiency in using a familiar tool really well.
     \item There's overhead from switching to the new thing. Even if
     it's better, the marginal improvement may be too low.
     \item New isn't always better.
     \item There's always a newer thing\footnote{Every time someone starts a new web framework, God kills a kitten.}.
   \end{itemize}  
   
   We can still try new things. Look for things that function poorly and try to replace those.
   It's not like they're in short supply.
   
\end{frame}


\end{document}
