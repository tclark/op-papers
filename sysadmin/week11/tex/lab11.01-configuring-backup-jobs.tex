\documentclass{article}   	% use "amsart" instead of "article" for AMSLaTeX format
\usepackage[margin=0.5in]{geometry}                		% See geometry.pdf to learn the layout options. There are lots.
\geometry{a4paper}                   		% ... or a4paper or a5paper or ...

\usepackage[parfill]{parskip}    		% Activate to begin paragraphs with an empty line rather than an indent
\usepackage{graphicx}				% Use pdf, png, jpg, or eps with pdflatex; use eps in DVI mode
\usepackage{enumerate}								% TeX will automatically convert eps --> pdf in pdflatex		


\title{Lab 11.2:  Configuring Backup Jobs\\ IN719 Systems Administration}
\date{}							% Activate to display a given date or no date

\begin{document}
\maketitle

\section*{Introduction}
So far we have set up Bacula and are prepared to configure it, but we have not configured it to do any real backups/restores. In this lab we will walk through doing this.

All of the backup and restore jobs are configured in the file \texttt{/etc/bacula/bacula-dir.conf} on \texttt{storage}. We also have to do some configuration in  \texttt{/etc/bacula/bacula-fd.conf} on each backup client machine, and a small bit of configuration in  \texttt{/etc/bacula/bacula-sd.conf} on \texttt{storage}.

\section{Configuring a Backup Job}
The details of a backup job are configured in a \texttt{Job} section in the configuration file.  You will have a \texttt{Job} section for each client that you back up.  If your backup jobs include a lot of shared configuration, you can create a \texttt{JobDefs} section that provides default values that can be applied to \texttt{Jobs}.  These defaults can be overridden if necessary.  The use of a \texttt{JobDefs} is optional.

A backup \texttt{Job} contains a few core components - that themselves have to be defined - and a handful of other values.  Important components of a backup \texttt{Job} are

\begin{itemize}
	\item \texttt{Client}: the client \texttt{bacula-fd} whose data is to be backed up;
	\item \texttt{FileSet}: a set of files and directories on the client machine to back up;
	\item \texttt{Schedule}: when to perform the backups;
	\item \texttt{Storage}: the storage daemon/device that will hold or backed up data.
\end{itemize}

All of these elements need their own configurations.

First, let's look at a \texttt{Client} configuration in \texttt{bacula-dir.conf}.

\begin{verbatim}
Client {
  Name = example-fd
  Address = 10.25.123.123
  FDPort = 9102
  Catalog = MyCatalog
  Password = "lolcats"          # password for FileDaemon
  File Retention = 60 days            # 60 days
  Job Retention = 6 months            # six months
  AutoPrune = yes                     # Prune expired Jobs/Files
}
\end{verbatim}

Note a few things here:
\begin{enumerate}
  \item The Name property need to match the Name that is set in the \texttt{bacula-fd.conf} file on the client machine.
  \item The Address can be an ip address or a hostname that resolves in DNS.
  \item The Password needs to match the Password set in the corresponding \texttt{bacula-fd.conf}. There are a lot of passwords
  in our configuration and it's hard to keep them straight. My advice is to use one easy password everywhere while you get things working,
  then go back and use a more secure password scheme later.
\end{enumerate}
 
\newpage

Now we need to look at an example \texttt{FileSet}.
  
  \begin{verbatim}
  FileSet {
    Name = "ExampleSet"
    Include {
      Options {
        signature = MD5
        Exclude = yes
       }
       File = /etc
       File = /home
   }
   Exclude {
       File = *.swp
   }
  }
  \end{verbatim}
  
This \texttt{FileSet} will back up everything under \texttt{/home} and \texttt{/etc} but will ignore vi swap files. Note that there is nothing 
here that is specific to any client machine, so we could use it on any one where we want to back up \texttt{/home} and \texttt{/etc}.
 
There is already a good example of a working \texttt{Schedule} definition in \texttt{bacula-dir.conf} that looks like this

\begin{verbatim}
Schedule {
  Name = "WeeklyCycle"
  Run = Full 1st sun at 23:05
  Run = Differential 2nd-5th sun at 23:05
  Run = Incremental mon-sat at 23:05
}
\end{verbatim}

Now we need a \texttt{Storage} device, but it turns out that we set one up in the first Bacula lab, called \texttt{File1}.
You will see that we refer to it in the \texttt{Job} definition below.


For the \texttt{Job}, there are a few values that are important:

\begin{itemize}
	\item \texttt{Name}: each backup job needs a distinct name;
	\item \texttt{Type}: jobs are either \texttt{Backup} or \texttt{Restore} jobs;
	\item \texttt{Level}: backup jobs can be \texttt{Full} or \texttt{incremental}.	
\end{itemize}

Here is an example of a backup \texttt{Job}:

\begin{verbatim}
Job {
  Name = "ExampleBackp"
  Type = Backup
  Level = Incremental
  Client = example-fd
  FileSet = "ExampleSet"
  Schedule = "WeeklyCycle"
  Storage = File1
  Messages = Standard
  Pool = File
  SpoolAttributes = yes
  Priority = 10
  Write Bootstrap = "/var/lib/bacula/%c.bsr"
}

\end{verbatim}

Finally, for any of this to work, we need the client side configured in \texttt{/etc/bacula/bacula-fd.conf} Rather that copy the full configuration here, we will just note the 
key fields.

\begin{itemize}
  \item In the \texttt{Director} section, make sure the \texttt{Name} and \texttt{Password} values match those in \texttt{bacula-dir.conf}.
  \item In the \texttt{FileDaemon} section, make sure the name matches the corresponding \texttt{Name} in the \texttt{Client} section
  in \texttt{bacula-dir.conf}.
  \item In the \texttt{FileDaemon} section, make sure the \texttt{Address} matches the IP address on which the file daemon should be listening.
\end{itemize}


\end{document}

