\documentclass{article}   	% use "amsart" instead of "article" for AMSLaTeX format
\usepackage[margin=0.5in]{geometry}                		% See geometry.pdf to learn the layout options. There are lots.
\geometry{a4paper}                   		% ... or a4paper or a5paper or ...

\usepackage[parfill]{parskip}    		% Activate to begin paragraphs with an empty line rather than an indent
\usepackage{graphicx}				% Use pdf, png, jpg, or eps with pdflatex; use eps in DVI mode
\usepackage{enumerate}								% TeX will automatically convert eps --> pdf in pdflatex		


\title{Lab 11.2:  Configuring Backup Jobs\\ IN719 Systems Administration}
\date{}							% Activate to display a given date or no date

\begin{document}
\maketitle

\section*{Introduction}
So far we have dealt with two parts of our Bacula setup:
\begin{enumerate}
	\item Installing the needed components on various servers;
	\item Configuring those components to communicate with each other over the network.
\end{enumerate}

But we're not backing up anything yet.  We will work on that now.

All of the backup and restore jobs are configured in the file \texttt{/etc/bacula/bacula-dir.conf} on \texttt{mgmt}.

\section{Configuring a Backup Job}
The details of a backup job are configured in a \texttt{Job} section in the configuration file.  You will have a \texttt{Job} section for each client that you back up.  If your backup jobs include a lot of shared configuration, you can create a \texttt{JobDefs} section that provides default values that can be applied to \texttt{Jobs}.  These defaults can be overridden if necessary.  The use of a \texttt{JobDefs} is optional.

A backup \texttt{Job} contains a few core components - that themselves have to be defined - and a handful of other values.  Important components of a backup \texttt{Job} are

\begin{itemize}
	\item \texttt{Client}: the client machine whose data is to be backed up;
	\item \texttt{FileSet}: a set of files and directories on the client machine to back up;
	\item \texttt{Schedule}: when to perform the backups;
	\item \texttt{Storage}: the storage daemon that will hold or backed up data.
\end{itemize}

All of these elements need their own configurations.

There are also a few values that are important:

\begin{itemize}
	\item \texttt{Name}: each backup job needs a distinct name;
	\item \texttt{Type}: jobs are either \texttt{Backup} or \texttt{Restore} jobs;
	\item \texttt{Level}: backup jobs can be \texttt{Full} or \texttt{incremental}.	
\end{itemize}

Here is an example of a backup \texttt{Job}:

\begin{verbatim}
Job {
  Name = "ExampleBackp"
  Type = Backup
  Level = Incremental
  Client = exampleserver-fd
  FileSet = "ExampleSet"
  Schedule = "WeeklyCycle"
  Storage = File
  Messages = Standard
  Pool = File
  Priority = 10
  Write Bootstrap = "/var/lib/bacula/%c.bsr"
}

\end{verbatim}

For this to work, we need to define some elements. We need a \texttt{Client} definition for \texttt{exampleserver-fd}. We need a \texttt{Storage} definition. We discussed these items earlier in the week.  We need a \texttt{Storage} definition that we also did last time.  The \texttt{WeeklyCycle} schedule definition is included in the default configuration.  You can use it as a model to create alternative \texttt{Schedule}s.
The default \texttt{File} pool will suffice.

We do need to look at an example \texttt{FileSet}.
  
  \begin{verbatim}
  FileSet {
    Name = "ExampleSet"
    Include {
      Options {
        signature = MD5
        Exclude = yes
       }
       File = /etc
       File = /home
   }
   Exclude {
       File = *.swp
   }
  }
  \end{verbatim}
  
This \texttt{FileSet} will back up everything under \texttt{/home} and \texttt{/etc} but will ignore vi swap files.

\end{document}

