\documentclass{article}
\usepackage{graphicx}
\usepackage{wrapfig}
%\usepackage{inconsolata}
\usepackage{enumerate}
\usepackage{hyperref}
\usepackage{verbatim}
\usepackage[parfill]{parskip}
\usepackage[margin = 2.5cm]{geometry}

\usepackage[T1]{fontenc}


\begin{document}

\title{Docker Volumes\\ IN720 Virtualisation}
\date{}
\maketitle

\section*{Introduction}
In this lab we are going to play around with Docker volumes to see how we can create and seed volumes with data that we can then share across other containers.


\section{Build a data container}
Build a new container from a Dockerfile that with the following:

\begin{enumerate}
	\item Create a directory, \texttt{/data} inside your container.
	\item In \texttt{/data}, create files named \texttt{foo}, \texttt{bar}, and \texttt{baz}.
	\item Make \texttt{/data} a volume.
\end{enumerate}

Build your image and launch a container called \texttt{vol\_lab\_1} from it.


\section{Find your volume on the host system}
In another ssh session, inspect your new container to find its volumes with the command

\texttt{sudo docker inspect -f \{\{.Volumes\}\} vol\_lab\_1}

Once you find the directory on the host that stores your volume, modify one of the files inside it.  Attach a terminal to \texttt{vol\_lab\_1} and see if the changes on the host system are visible inside the container.

\section{Using the volume in another container}
Shut down \texttt{vol\_lab\_1}.  Start a new container with the command

\texttt{sudo docker run -it  --volumes-from vol\_lab\_1 ubuntu /bin/bash}

When your new container starts, inspect the \texttt{/data} volume inside it.  Make some changes inside the volume and then shut down this container.  Then, check the volume directory on the host system and see if your changes are visible there.
\end{document}