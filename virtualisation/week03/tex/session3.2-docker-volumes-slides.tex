% Beamer slide template prepared by Tom Clark <tom.clark@op.ac.nz>
% Otago Polytechnic
% Dec 2012

\documentclass[10pt]{beamer}
\usetheme{Dunedin}
\usepackage{graphicx}
\usepackage{fancyvrb}

\newcommand\codeHighlight[1]{\textcolor[rgb]{1,0,0}{\textbf{#1}}}

\title{Docker Volumes}

\author[IN720]{Virtualisation}
\institute[Otago Polytechnic]{
  Otago Polytechnic \\
  Dunedin, New Zealand \\
}
\date{}
\begin{document}

%----------- titlepage ----------------------------------------------%
\begin{frame}[plain]
  \titlepage
\end{frame}


%----------- slide --------------------------------------------------%
\begin{frame}
  \frametitle{Union File Systems}

 \begin{itemize}
  \item Files in a Docker container are presented in a \emph{Union File System}
  \item An image is stored as a set of read-only layers. 
  \item Only the very top layer in a running container is read-write.
  \item When we modify a file from an image, we copy it from a lower, read-only layer into the read-write layer.
  \item The modified copy saved on the top layer masks the unmodified version on the lower layer.
  \item When we destroy a container, the read-write layer associated with it is destroyed and our changes are lost.
  
  \end{itemize}
\end{frame}

\begin{frame}
	\frametitle{Docker Volumes}
	\begin{itemize}
		\item Sometimes we want to persist data created inside a container, or we want to share data between containers.
		\item A Docker \emph{volume} lets us do this.
		\item Volumes are files or directories that sit outside the union file system and are saved directly on the host file system.
		\item Other containers can access volumes, and we can delete a container without deleting its volumes.
	\end{itemize}
\end{frame}

\begin{frame}[fragile]
	\frametitle{Creating Volumes}
	We can create volumes from the command line:
	
	\begin{Verbatim}[commandchars=\\\{\}]
	sudo docker run -it --name vol-test \codeHighlight{-v /data} ubuntu /bin/bash
	\end{Verbatim}
\end{frame}

\begin{frame}[fragile]
	\frametitle{Creating Volumes}
	We can create volumes from the in a Dockerfile:
	
	\begin{Verbatim}[commandchars=\\\{\}]
	FROM ubuntu:14.04
	\codeHighlight{VOLUME /data}
	\end{Verbatim}
\end{frame}

\begin{frame}[fragile]
	\frametitle{Creating Volumes}
	We can mount volumes that are associated with another container:
	
	\begin{Verbatim}[commandchars=\\\{\}]
	docker run -it --name vol-test-2 
	\codeHighlight{--volumes-from vol-test} 
	ubuntu /bin/bash
	\end{Verbatim}
	
	Note that 
	\begin{itemize}
		\item The container \texttt{vol-test} \emph{doesn't have to be running}.
		\item Changes to the volume are immediately visibile in both containers (and the host).
	\end{itemize}
\end{frame}

\begin{frame}[fragile]
	\frametitle{Creating Volumes}
	We can create volumes from directories that already exist on the host:
	
	\begin{Verbatim}[commandchars=\\\{\}]
	sudo docker run -it --name vol-test3 
	\codeHighlight{-v /home/tclark/data:/data} 
	ubuntu /bin/bash
	\end{Verbatim}
\end{frame}


\begin{frame}
	\frametitle{Data-only containers}
	
	One application of containers with volumes is creating containers with the express purpose of establishing data volumes that are accessed by other containers.  These are called \emph{data-only containers} and they are described at \\
	 \mbox{\url{http://container42.com/2014/11/18/data-only-container-madness/}}
\end{frame}

\end{document}