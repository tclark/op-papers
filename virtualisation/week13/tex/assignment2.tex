\documentclass{article}
\usepackage{graphicx}
\usepackage{wrapfig}
%\usepackage{inconsolata}
\usepackage{enumerate}
\usepackage{hyperref}
\usepackage{verbatim}
\usepackage[parfill]{parskip}
\usepackage[margin = 2.5cm]{geometry}

\usepackage[T1]{fontenc}


\begin{document}

\title{Assignment 2: OpenStack \\ IN720 Virtualisation}
\date{}
\maketitle

\section*{Introduction}
In this assignment you will create an OpenStack project (or \emph{tenant}) and then create and launch a virtual machine instance in it.

This assignment is worth 25\% of your mark in this paper.

\section{Tasks}

\begin{enumerate}
	\item Create an OpenStack tenant and a user who is a member of the tenant.
	\item Load an Ubuntu 14.04 image for the tenant from the Canonical image collection at \url{http://cloud-images.ubuntu.com/}.
	\item Create a 1GB data volume for the tenant in our OpenStack block storage service.
	\item Launch a virtual machine image in the tenant.  Mount and format the volume from block storage in the virtual machine 
	and save a file on it.  Then detach the image from the VM instance and shut down the VM.
	\item Launch a second VM instance in the tenant and mount the storage volume in it.  Verify that yopur saved file is present.
	\item Set up a floating IP address for the tenant and associate it with the second VM instance.
	\item Set the security group settings for the second virtual machine to allow ssh access from any client.
	\item Document your work step-by-step and submit a clear, easy-to-follow HOWTO document, in PDF format, to the lecturer.  Your target audience for the HOWTO is a typical student who has completed the first year of the BIT.
	\item Collaborating with the rest of the class, troubleshoot our OpenStack platform as necessary.
\end{enumerate}

\section{Submission}
This assignment is due on Friday, 13 November. Your submission will include the artifacts on our OpenStack cloud as described above and the HOWTO 
document.  70\% of the assignment mark is based on the OpenStack tasks and 30\% is based on the supporting documentation. 


\end{document}