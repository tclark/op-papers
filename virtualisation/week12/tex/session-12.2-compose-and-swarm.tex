% Beamer slide template prepared by Tom Clark <tom.clark@op.ac.nz>
% Otago Polytechnic
% Dec 2012

\documentclass[10pt]{beamer}
\usetheme{Dunedin}
\usepackage{graphicx}
\usepackage{fancyvrb}

\newcommand\codeHighlight[1]{\textcolor[rgb]{1,0,0}{\textbf{#1}}}

\title{Compose and Swarm}

\author[I720]{Virtualisation}
\institute[Otago Polytechnic]{
  Otago Polytechnic \\
  Dunedin, New Zealand \\
}
\date{}
\begin{document}

%----------- titlepage ----------------------------------------------%
\begin{frame}[plain]
  \titlepage
\end{frame}

\begin{frame}
  \frametitle{Part 1: Docker swarm}
  
  We've just been working on one Docker server. However, Docker is 
  a service that listens on network interfaces. New versions have the 
  capability to be joined with other docker servers over a network to
  form a \emph{swarm}.
\end{frame}

\begin{frame}
  \frametitle{Forming a swarm}
  
  \begin{enumerate}
    \item Initialise the swarm on one server. This server becomes a \emph{manager}.
    \item Optionally, join additional managers to the swarm\footnote{Best practice is to limit the number of managers to 3-7.}.
    \item Join \emph{workers} to the swarm.
    \item On a manager node we may issue commands to control and monitor the swarm.
  \end{enumerate}  
\end{frame}

\begin{frame}
  \frametitle{Part 2: Compose and swarm}
  
  There are a few options for deploying containerised services on a swarm.
  We will see how to use a Docker Compose file for this.
\end{frame}


\begin{frame}[fragile]
  \frametitle{To some extent, it just works}
  
  Make a \texttt{docker-compose.yml} file, then run
  
  \begin{verbatim}
  $ docker-compose up
   \end{verbatim}
   
   There are some issues, however\footnote{You don't actually do it this way, though.}.
\end{frame}

\begin{frame}
  \frametitle{First issue, building}
   
   \begin{itemize}
     \item We saw that you can specify that Docker build an image in a Compose file.
     \item But swarm needs images to be accessible in a registry.
     \item So, building in a compose file used for swarm deployment is right out.
     \item Really, when you're deploying to a production setting, you shouldn't be building on the fly anyway.
   \end{itemize}
\end{frame}

\begin{frame}
  \frametitle{Next issue, volumes}
   
   \begin{itemize}
     \item We use volumes to let containers work directly with the host file system.
     \item But we don't always know what host our containers will run upon.
     \item If our container creates volumes, use named ones.
     \item If our containers read from volumes, have other containers that populate them, then use a \texttt{volumes\_from} directive.
   \end{itemize}
\end{frame}

\begin{frame}
  \frametitle{Issue three: dependencies}
   
   The \texttt{depends\_on} directive in a compose file has a different meaning in a swarm context.
   \begin{itemize}
     \item We can create dependencies between containers
              \begin{itemize}
                \item Explicitly: \texttt{depends\_on}
                \item Implicitly: \texttt{volumes\_from}
              \end{itemize}
         
     \item In either case, Docker interprets this to mean that a container must be deployed on the same node as its dependencies.
     \item In simple cases this works fine.
     \item We need to be aware of this when scaling services. When scaling a service, we may need to scale its dependencies.
   \end{itemize}
\end{frame}


\begin{frame}[fragile]
    \frametitle{The tricky case: multiple dependencies}
    
    \begin{verbatim}
version: "3.7"
services:
    svc_a:  
        image: image_a
        depends_on:
            - svc_b
            - svc_c
    svc_b:  
        image: image_b
    svc_c:  
        image: image_c
    \end{verbatim}
    
    In this case, \texttt{svc\_a} must be deployed on a host that is already running \texttt{svc\_b} and \texttt{svc\_c}. But we aren't guaranteed that 
    \texttt{svc\_b} and \texttt{svc\_c} will be placed together on the same host. There is a \texttt{constraints} directive available in Compose that helps deal with this.
\end{frame}

\begin{frame}[fragile]
    \frametitle{Replicas}
    
    We will see in the lab that we can run multiple replicas of a container on a swarm. We control this
    by setting a \texttt{replicas} value in a \texttt{deploy} section.
    
    \begin{verbatim}
    
    services:
      foo:
        image: tclark/foo
        deploy:
          replicas: 3
    

    \end{verbatim}
    
   
\end{frame}


\begin{frame}
  \frametitle{Final issue: networks}
   
   \begin{itemize}
     \item We've seen that Docker can set up internal networks that allow containers to communicate and find each other by name.   
     \item What happens when the containers are on seperate hosts?
     \item We can get the same result by creating \emph{overlay networks}.
   \end{itemize}
\end{frame}

\begin{frame}[fragile]
    \frametitle{In compose}
     \begin{verbatim}
networks:
     foo:
         driver: overlay
     \end{verbatim}
 \end{frame}
 
 \begin{frame}[fragile]
    \frametitle{Running your services}
     \begin{verbatim}
     
docker stack deploy --compose-file <path to file> stack-name
     
     \end{verbatim}
 \end{frame}
 
 \begin{frame}
  \frametitle{Final caveats}
   
   \begin{itemize}
     \item This stuff is highly version-dependent.
     \item Documentation is sometimes lacking and what's there can be confusing.
     \item Stack Overflow does not care about you.
   \end{itemize}
   
   Besides the Docker CLI and Compose File references on \url{https://docs.docker.com/}, the docs at
   \url{https://docs.docker.com/engine/swarm/swarm-tutorial/} are useful for the information we looked at
   here as well as other topics.
   
   
   
   \end{frame}    

\end{document}
