\documentclass{article}
\usepackage{graphicx}
\usepackage{wrapfig}
\usepackage{inconsolata}
\usepackage{enumerate}
\usepackage{hyperref}
\usepackage{verbatim}
\usepackage[parfill]{parskip}
\usepackage[margin = 2.5cm]{geometry}

\usepackage[T1]{fontenc}


\begin{document}

\title{Password Hashing\\ IN618 Security}
\date{}
\maketitle

\section*{Introduction}
In your future work, the probability that you will be responsible for handling user names and passwords is basically 100\%
In this lab you will start learning how to do this correctly.

\section{Examining a trivially exploitable system}
Extract a copy of the IN618-Password-Example-One project onto your lab machines desktop.  Open the project 
with Visual Studio, inspect the source code, and run the program.  It implements a very simple username/password 
check and allows you to enter new usernames and passwords.

\textbf{Problem 1:} There are several username/password pairs already set up on the system.  By inspecting it,
find them and verify that you are able to log into the system using stolen credentials.  Write one username/password
pair that you found.

\vspace{25mm}

Why was it so easy to do this?

\newpage      

\section{Examining a more secure, but still flawed, system}
After the password hashing method is discussed, implement it on your system. and verify that it works.  Then, get a copy 
of the pre-hashed passwords and set them up on your system.

\textbf{Problem 2:} One of the passwords in the pre-hashed set is poorly chosen and can be discovered using the system.
Which user has the bad password and what is it?

\vspace{25mm}

Briefly describe the method you used to find the password.  Given enough time, could you use it to find all 
the passwords in the file?  

\vspace{200mm}

\textbf{Challenge question:} Without consulting any references, can you think of a way to make this 
password system harder to break? How?


\vspace{50mm}

\section{Wrapping up}
Save a copy of your project in your home directory or other location where you can save it for the next lab session.
\end{document}
