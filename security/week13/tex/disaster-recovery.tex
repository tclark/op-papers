% Beamer slide template prepared by Tom Clark <tom.clark@op.ac.nz>
% Otago Polytechnic
% Dec 2012

\documentclass[10pt]{beamer}
\usetheme{CambridgeUS}
\usepackage{graphicx}
\usepackage{fancyvrb}


\title{Disaster Recovery}

\author[IN618]{Introduction to IT Security}
\institute[Otago Polytechnic]{
  Otago Polytechnic \\
  Dunedin, New Zealand \\
}
\date{}
\begin{document}

%----------- titlepage ----------------------------------------------%
\begin{frame}[plain]
  \titlepage
\end{frame}



%----------- slide --------------------------------------------------%
\begin{frame}
  \frametitle{What do we mean by disaster?}

 By \emph{disaster}, we mean an event that stops or severely hinders
 an organisation from conducting its business.

 \begin{itemize}
  \item Flood, earthquake, etc.
  \item Building fire
  \item Major security breach
  \item Network or power outage
 \end{itemize}
\end{frame}

Since there are many types of disasters, there is no single solution
to the problem of disaster response and recovery.

%----------- slide --------------------------------------------------%
\begin{frame}
  \frametitle{You must have DR plans}


 \begin{itemize}
  \item DR planning is an organisation-wide issue
  \item ICT assets are typically a critical (and difficult) component of DR plans
 \end{itemize}
\end{frame}


%----------- slide --------------------------------------------------%
\begin{frame}
  \frametitle{First steps: risk analysis}


 \begin{itemize}
  \item Identify possible disasters of concern 
  \item Prioritise
  \item Set budget
 \end{itemize}
\end{frame}




%----------- slide --------------------------------------------------%
\begin{frame}
  \frametitle{Next steps: risk reduction}

 It is generally very cost-effective to take steps to reduce the risks of disasters
 striking in the first place.

 \begin{itemize}
  \item Building planning
  \item Fire supression
  \item Earthquake strengthening
  \item Physical and network security
  \item Power conditioning and supply
  \item Network redundancy
  \item Deal with well-prepared suppliers and service providers
 \end{itemize}
\end{frame}



%----------- slide --------------------------------------------------%
\begin{frame}
  \frametitle{Next steps: damage mitigation}

 We can't completely prevent disasters, so we plan to reduce the harm
 caused when they occur.
 
 \begin{itemize}
  \item Backup your data
  \item Prepare a list of needed hardware and have a plan to get it
  \item Prepare a backup location(s) with adequate space, power, and connectivity
  \item Have the right service contracts in place
  \item Make sure that your people know the plan
 \end{itemize}
\end{frame}



%----------- slide --------------------------------------------------%
\begin{frame}
  \frametitle{More on backups}


 \begin{itemize}
  \item Reliable backups are critical to a wide range of DR scenarios
  \item You must move your backed-up data to a secure offsite location
  \item If you need special hardware to restore from backup (e.g., a high-end 
	  tape drive), then you need that hardware in secure offsite location.
  \item There are a number of service providers who offer reasonably priced 
	  backup service
  \item You must document your restoration procedure and test it regularly.
 \end{itemize}
\end{frame}

%----------- slide --------------------------------------------------%
\begin{frame}
  \frametitle{Practice}


 The final step in DR preparation is to conduct drills in which you carry out
 you DR plans.

 This can, and should, span a range of activities from table top simulations to
 full scale rehearsals.
\end{frame}


\end{document}

