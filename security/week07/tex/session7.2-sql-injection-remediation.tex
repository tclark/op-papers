\documentclass{article}
\usepackage{graphicx}
\usepackage{wrapfig}
\usepackage{inconsolata}
\usepackage{enumerate}
\usepackage{hyperref}
\usepackage{verbatim}
\usepackage[parfill]{parskip}
\usepackage[margin = 2.5cm]{geometry}

\usepackage[T1]{fontenc}


\begin{document}

\title{SQL Injection Remediation\\ IN618 Security}
\date{}
\maketitle

\section*{Introduction}
In this lab we will take an application with SQL injection vulnerabilities and 
modify it to remove them.  From the \texttt{I:} drive, download and extract
the file \texttt{SQLInjectionRemediation.zip}.  It contains a Visual Studio 
project that we will use for this exercise.

Recall that there are four things we identified last time that reduce
vulnerability to SQL injections:

\begin{enumerate}
	\item sanitise inputs;
	\item use parameterised queries;
	\item set correct DBMS permissions;
	\item control unecessary error message output.
\end{enumerate}

Of these, we will focus on the first two.  Our SQLite database doesn't
have a strong permissions model and our error messages would be handled
when we deploy a production version of our application.

\section{Verify the vulnerability}
This application has the same SQL injection vulnerability that the example
application we saw last time does. We can verify this by insepcting the source code
and by trying the exploit. The easiet way to exploit this application is by inputting 
your attack code in the ``Last Name'' text box and performing a search.

\section{Sanitise your inputs}
Initially, the application makes no effort to sanitise its inputs before
using them in database queries.  Sanitising inputs for SQL is a bit hard, but there
are some things we can do:

\begin{itemize}
	\item Limit the length of input strings to appropriate values.  You 
		can use the C\# \texttt{Substring} method for this.
	\item Remove inappropriate characters from the inputs.  You can
		use the C\# \texttt{Remove} or \texttt{Replace} methods
		for this.
\end{itemize}

\newpage

\section{Parameterised queries}
Last time we discussed parameterised queries.  Here is an example of
a parameterised query prepared for SQLite in C\#:

\begin{verbatim}

SQLiteCommand command = new SQLiteCommand(db_connection);
string sql = "select * from items where item_type=@type";
command.CommandText = sql;
command.Prepare();
command.Parameters.AddWithValue("@type", "B");
SQLiteDataReader reader = command.ExecuteReader();
\end{verbatim}

This code will find items whose \texttt{type} is ``\texttt{B}''.

Using this technique, modify your application to use parameterised queries.

\section{Submission}
Commit your modified code to your SVN repository before the next class meeting.  To show
that you are using SVN properly, your commit history for this exercise should have at least 
three commits: one when you begin work before modifying any code, one when you have added input
sanitisation, and one when you add parameterised queries.

\end{document}
