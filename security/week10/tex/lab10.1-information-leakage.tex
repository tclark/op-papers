\documentclass{article}
\usepackage{enumerate}
\usepackage{verbatim}
\usepackage[parfill]{parskip}
\usepackage[margin = 2.5cm]{geometry}

\usepackage[T1]{fontenc}


\begin{document}

\title{ Lab 10.1: Information Leakage \\ IN618 Security}
\maketitle

\section*{Introduction}
In the processes of carrying out their normal functions it is inevitable that our systems will  expose
some information about the software they are running. Sometimes this is necessary and sometimes it can't
be avoided. But this inforamtion may be of use to attackers, so it's important that we make deliberate
choices about what information we seek to disclose.

In this lab we will see two ways that we can gather information about software that is running on a
server.

\section{Inspecting web headers}
We know that web servers send response headers in addition to their payloads with each response. These
headers typically contain information about the web server software being used.  We can inspect 
these headers using the text-based \texttt{lynx} browser.  Log onto \texttt{sec-student.sqrawler.com}
and run these commands:

\begin{verbatim}
  lynx --head http://www.op.ac.nz

  lynx --head http://www.otago.ac.nz
\end{verbatim}

By inspecting the headers, and perhaps by searching the web a bit, answer the following questions 
about each we server:

\begin{enumerate}
	\item What kind of web server is the site running?
	\item What other software or services are used to deliver the site?
\end{enumerate}

\vspace{500mm}

We get a limited amount of information from this method and we can't even be sure that it's accurate.
But we may still learn something that will help us plan an attack.  Since we're just making routine
web requests, there's nothing about our actions so far that warn system operators of a possible attack.

\section{Port scanning}
There are other scanning methods that can gather more information.  \emph{Port scanning} is one 
such technique.  Although it gathers more and better information for an attacker, port scans
can be detected by system operators and should be regarded as the first stage of an intrusion 
attempt. \textbf{Do not perform invasive port scans of systems without the consent of their
operators.  In some cases you may be liable for criminal prosecution.}

For this lab you may perform scans of the systems set up for this exercise.  The lecturer will 
indicate the IP addresses of servers you may scan. You will also be given the address of a system that
you will \texttt{ssh} into to perform these scans.

\begin{verbatim}
  nmap -sv --version-light <target ip address>

  nmap -sv --version-all <target ip address>
\end{verbatim}

The first scan is quicker but yields less information.  The second scan takes a long time and produces
a more thourough report. Scan one or two of the sample addresses and note your findings below.




\end{document}
