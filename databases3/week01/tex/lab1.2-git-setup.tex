\documentclass{article}
\usepackage{graphicx}
\usepackage{wrapfig}
\usepackage{hyperref}
\usepackage{inconsolata}
\usepackage{enumerate}
\usepackage{verbatim}
\usepackage[parfill]{parskip}
\usepackage[margin = 2.5cm]{geometry}

\usepackage[T1]{fontenc}


\begin{document}

\title{Lab 1.2: Git Setup\\ IN705 Databases Three}
\date{}
\maketitle

\section*{Introduction}
We will use Git and GitHub for version control in this paper.  In this lab we will walk through the setup of our Git repositories.  The book \emph{Pro Git} (listed in the resources section of the course directive) is a useful reference.

Git is already installed on your EC2 server.  You will need to log into that server to complete the tasks below.

\section{Setting Git preferences}
On your EC2 server, execute the following:

\begin{verbatim}
  $ git config --global user.name "Your Name"
  $ git config --global user.email "your email address"
\end{verbatim}

The email address you use should be the one associated with your GitHub account.


\section{Setting up ssh keys}
You will use ssh to transfer your code to and from GitHub.  To make this easier, you need to set up ssh keys.  This process is documented well at \url{https://help.github.com/articles/generating-ssh-keys}.  Follow the instructions there.

Note that the section mentioning xclip is specific to GNU/Linux desktops (and probably Macs). Copy and paste the key text in the manner appropriate to your desktop system.   

\section{Create a repository on GitHub}
You will keep all of your code for tis paper in one Git repository that we wil l name ``db3".  To create this repository, log onto GitHub and go to \url{https://github.com/new}.  Fill out the form there to create a public repository.  Don't tick the box to add a README, License, or .gitignore file.  We will add those files later.

\section{Initialise your local repository and push it to GitHub}
Log onto your EC2 server and execute the following:

\begin{verbatim}
 $ cd
 $ mkdir db3
 $ mv work/demo db3
 $ cd db3
 $ touch README.md
 $ git init
 $ git add .
 $ git commit -m "initial commit"
 $ git remote add origin git@github.com:your-git-username/db3.git
 $ git push -u origin master
\end{verbatim}

This assumes that you placed your hello world application in a directory named work.  Briefly here's what you did:  First, you organised your code into a directory named ``db3".  Inside that directory you created an empty README file.  Then you initialised your Git repository, committed your code to it, and then added a reference from your local repository to you GitHub repository.  Finally, you uploaded your code to GitHub.

Now you have a repository to store and track your work.  You should make frequent commits to this repository - one or more for each working day.  All of your code for this paper should be tracked in your GitHub repository.

\section{Wrapping up}
Email the lecturer with the URL for your GitHub repository.
\end{document}
