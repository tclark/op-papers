% Beamer slide template prepared by Tom Clark <tom.clark@op.ac.nz>
% Otago Polytechnic
% Dec 2012

\documentclass[10pt]{beamer}
\usetheme{CambridgeUS}
\usepackage{graphicx}
\usepackage{fancyvrb}


\title{Database Application Development}

\author[IN705]{Databases Three}
\institute[Otago Polytechnic]{
  Otago Polytechnic \\
  Dunedin, New Zealand \\
}
\date{}
\begin{document}

%----------- titlepage ----------------------------------------------%
\begin{frame}[plain]
  \titlepage
\end{frame}

%----------- slide --------------------------------------------------%
\begin{frame}
  \frametitle{The basic problem}
  We want to model something in the ``real world".
 \begin{itemize}
  \item Television programmes
  \item Ski fields
  \item Job candidates
  \end{itemize}
\end{frame}

%----------- slide --------------------------------------------------%
\begin{frame}
  \frametitle{Databases}
 \begin{itemize}
  \item A database is a good choice for storing our model data in many cases.
  \item Usually, this means a \emph{relational} database.
  \item Relational databases are well understood and there are plenty of
        excellent tools available for working on them.
  \item A properly designed and implemented relational database helps 
        guarantee the consistency and integrity of its data.
  \end{itemize}
\end{frame}

%----------- slide --------------------------------------------------%
\begin{frame}
  \frametitle{Database applications}
 \begin{itemize}
  \item Users aren't interested in writng SQL - nor should they be.
  \item We write database applications that provide our users with access to 
        the database.
  \item CRUD
     \begin{itemize}
       \item Create
       \item Read
       \item Update
       \item Delete
     \end{itemize}
  \item This is one of the main objectives of this paper.
  \end{itemize}
\end{frame}


%----------- slide --------------------------------------------------%
\begin{frame}
  \frametitle{Problem One}
  There is a fundamental impedance mismatch between relational data
  modeling and object-oriented programming.
\end{frame}

%----------- slide --------------------------------------------------%
\begin{frame}
  \frametitle{Solution One}
  Object-Relational Mapping (ORM) libraries help to resolve\footnote{Or perhaps they merely conceal it.} this difficulty.
\end{frame}

%----------- slide --------------------------------------------------%
\begin{frame}
  \frametitle{Problem Two}
  We often want to have multiple user applications that access the same 
  database.
  \begin{itemize}
   \item This makes problem one harder.
   \item We want a consistent interface to the data.
   \item We want modular code.
  \end{itemize}
\end{frame}


%----------- slide --------------------------------------------------%
\begin{frame}
  \frametitle{Problem Three}
  Is it webscale?\footnote{http://www.mongodb-is-web-scale.com} 
\vskip 1\baselineskip

  RDBMSs are difficult to scale up.  You can only go so far scaling vertically,  and scaling horizontally by clustering or partioning makes application 
  development even harder.
\end{frame}

%----------- slide --------------------------------------------------%
\begin{frame}
  \frametitle{Solution Two}
  We can build an integration application between the data store and client applications.
\end{frame}

%----------- slide --------------------------------------------------%
\begin{frame}
  \frametitle{Solution Three}
  Alternative database types
  \begin{itemize}
   \item Document stores
   \item Key-value stores
   \item Column-family stores
   \item Graph databases
  \end{itemize}

  These are commonly referred to collectively as \emph{NoSQL} databases.
\end{frame}

\end{document}
