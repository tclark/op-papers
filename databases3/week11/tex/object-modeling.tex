% Beamer slide template prepared by Tom Clark <tom.clark@op.ac.nz>
% Otago Polytechnic
% Dec 2012

\documentclass[10pt]{beamer}
\usetheme{CambridgeUS}
\usepackage{graphicx}
\usepackage{fancyvrb}
\usepackage{ulem}


\title{Object Modeling with Riak}

\author[IN705]{Databases Three}
\institute[Otago Polytechnic]{
  Otago Polytechnic \\
  Dunedin, New Zealand \\
}
\date{}
\begin{document}

%----------- titlepage ----------------------------------------------%
\begin{frame}[plain]
  \titlepage
\end{frame}



%----------- slide --------------------------------------------------%
\begin{frame}
  \frametitle{Previous versions}

  \begin{itemize}
	  \item Relational DB/ActiveRecord:  A lot of the db structure
		  was basically determined for us.
	  \item MongoDB/Mongoid:  The db structure was less fixed
		  but the tools still storngly influenced our
		  data modeling.
	  \item In both cases libraries handeled much of the data
		  storage and retrieval.
  \end{itemize}


\end{frame}



%----------- slide --------------------------------------------------%
\begin{frame}
  \frametitle{New version: Riak}

 \begin{itemize}
	 \item  Riak itself doesn't impose any db structure.
	 \item  Libraries available:  \sout{ripple}, riak-client
	 \item  riak-client handles db interactions, but doesn't do object modeling, so we're on our own.
 \end{itemize}

\end{frame}



%----------- slide --------------------------------------------------%
\begin{frame}
  \frametitle{Objects}
  

 \begin{itemize}
  \item User
	  \begin{itemize}
		  \item email
		  \item name
		  \item password
		  \item list of splatts
		  \item list of followed users
		  \item list of followers
	  \end{itemize}
  \item Splatt
	  \begin{itemize}
		  \item created-at timestamp
		  \item body
	  \end{itemize}
 \end{itemize}

\end{frame}


%----------- slide --------------------------------------------------%
\begin{frame}
  \frametitle{Relevant questions}

 \begin{itemize}
  \item What are the keys?
  \item How shall we store the records?  What are the buckets?
 \end{itemize}

\end{frame}


%----------- slide --------------------------------------------------%
\begin{frame}[fragile]
  \frametitle{Some sample code}

 \begin{verbatim}
require 'hashie'

class User < Hashie::Dash
  property :name
  property :password
  property :blurb
  property :follows
  property :followers
end
 \end{verbatim}

\end{frame}



%----------- slide --------------------------------------------------%
\begin{frame}[fragile]
  \frametitle{}

 \begin{verbatim}
 class UserRepository
   BUCKET = 'users'

   def initialize(client)
     @client = client
   end

   def save(user)

   end
   def get(user_name)

   end
 end

 \end{verbatim}

\end{frame}

\end{document}
