\documentclass{article}
\usepackage{graphicx}
\usepackage{wrapfig}
\usepackage{inconsolata}
\usepackage{hyperref}
\usepackage{enumerate}
\usepackage{verbatim}
\usepackage[parfill]{parskip}
\usepackage[margin = 2.5cm]{geometry}

\usepackage[T1]{fontenc}


\begin{document}

\title{Milestone 1: REST API Server\\ IN705 Databases Three}
\date{}
\maketitle

\section*{Introduction}
At this point our REST/JSON API server has nearly all of its required functionality and we are nearing our first milestone.  During the next week you will complete your API server and submit it for review.  

This is the first phase of your project assessment for this paper and it constitutes 20\% of your mark on the project.

\section{API server requirements}
Your API server support the following methods with the specified HTTP requests:

\begin{itemize}
 \item User methods
  \begin{itemize}
    \item \texttt{GET /users}  $\Rightarrow$ returns a list of all users
    \item \texttt{POST /users}  $\Rightarrow$ creates a new user with data in the POST body
    \item \texttt{PUT /users}  $\Rightarrow$ updates a user with data in the PUT body
    \item \texttt{GET /users/[:id]} $\Rightarrow$ returns the user with the specified id
    \item \texttt{DELETE /users/[:id]} $\Rightarrow$ deletes the user with the specified id
    \item \texttt{GET /users/splatts/[:id]}  $\Rightarrow$ returns a list of the specified user's splatts
    \item \texttt{GET /users/splatts-feed/[:id]}  $\Rightarrow$ returns the specified user's ``news feed''
    \item \texttt{POST /users/follows}  $\Rightarrow$ creates a follower/followed relationship between specified users
    \item \texttt{GET /users/follow/[:id]}  $\Rightarrow$ returns a list of the users followed by the specified user
    \item \texttt{GET /users/follower/[:id]}  $\Rightarrow$ returns a list of the users who follow the specified user
    \item \texttt{GET /users/splatts/[:id]}  $\Rightarrow$ returns a list of the specified user's splatts
    \item \texttt{DELETE /users/[:id1]/[:id2]} $\Rightarrow$ causes user with id1 to unfollow the user with id2

  \end{itemize}

 \item Splatt methods
  \begin{itemize}
    
    \item \texttt{GET /splatts/[:id]} $\Rightarrow$ returns the splatt with the specified id
    \item \texttt{POST /splatts}  $\Rightarrow$ creates a new splatt with data in the POST body
    \item \texttt{DELETE /splatts/[:id]} $\Rightarrow$ deletes the splatt with the specified id
  \end{itemize}
\end{itemize}

\section{Testing script}
  You also need to supply a simple shell script with the \texttt{curl} commands that do the following in the specified order.
  \begin{enumerate}
     \item create 3 users
     \item create 5 splatts for each user in a manner that mixes the chronological order of the splatts between the 3 users
     \item causes the first user to follow the other 2
     \item gets the first user's splatts
     \item gets the users followed by the first user 
     \item gets the first user's news feed
     \item causes the first users to unfollow the third user
     \item displays the first user's news feed after unfollowing the third
  \end{enumerate}

\section{Submission requirements}
  You will submit your source code and testing script by \emph{tagging}\footnote{See \url{http://git-scm.com/book/Git-Basics-Tagging}.  Use a lightweight or annotated tag labeled ``M1''.} a commit to your GitHub repository with the tag ``M1''.  This commit must be dated no later than Monday, 18 August at 18:00 NZST. Late submissions will be penalised 10\% per day beginning at 18:00:01 each day beginning on 18 August.  Improper submissions will not be marked.  

To mark your submission, the lecturer shall clone your GitHub repository at the tagged point, set up the appliation (i.e., run \texttt{rake db:migrate}), run your test script and other curl commands, and inspect your source code.  Your submission will be marked according the the following schedule.

\begin{itemize}
  \item correct and complete implementation of the API:  50\%
  \item test script runs and produces correct results: 10\%
  \item quality of source code: 40\%
\end{itemize}
\end{document}
