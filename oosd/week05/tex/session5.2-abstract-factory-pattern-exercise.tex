\documentclass{article}
\usepackage{graphicx}
\usepackage{wrapfig}
\usepackage{inconsolata}
\usepackage{enumerate}
\usepackage{hyperref}
\usepackage[margin = 2.25cm]{geometry}



\begin{document}


\title{Abstract Factory Pattern Exercise\\IN710 Object Oriented System Development}
\date{}
\maketitle

\section*{Introduction}
The \emph{Abstract Factory Pattern} can be used to create factories 
that produce families of related objects.

In this exercise you will implement an abstract factory pattern 
to represent postal addresses and phone numbers in the USA and the
UK.

section{The problem}
We want to model personal contact information for people in various counries,
for now we'll focus on the US and UK.  We will store and represent postal addresses
and phone numbers.  This means that our classes will need to adapt to the 
data and display formats of the two countries.  When we create a US contact, 
we want a US address and a US phone number.  For UK contacts, we will use UK 
addresses and phone numbers.  This mean that right now we have four classes 
in two groups on our hands.

\begin{itemize}
	\item US addresses and US phone numbers
	\item UK addresses and UK phone numbers
\end{itemize}

We need to create the right set of objects at runtime based on the 
locations of particular contacts.  This is the kind of situation
where the \emph{Abstract Factory Pattern} works well.

\section{The task}
You need to define \texttt{AbstractPhoneNumber} and 
\texttt{AbstractPostalAddress} classes.  classes.  From these, derive 
\texttt{UKPhoneNumber}, \texttt{USPhoneNumber}, \texttt{UKAddress},
and \texttt{USAddress} Classes.  These derived classes will validate
their data and need to supply \texttt{print} methods that print their
data in the correct formats.

Then, define an \texttt{AbstractContactFactory} class that provides
two public methods, \texttt{create\_phone\_number} and 
\texttt{create\_address}.  From this, derive the classes 
\texttt{UKContactFactory} and \texttt{USContactFactory} that implement
create methods that return the right kinds of contact objects.

Don't forget to write appropriate unit tests.

\subsection{Abstract classes in Python}

Abstract bases classes are not often used in Python.  In the absence
of compile-time type checking they are less important than they are in language
like Java or C++.  But they still have some value, in part in defining the
interface derived classesa are expected to support. There are a couple of ways 
you could define an abstract class.

Example 1:

\begin{verbatim}

class FooClass:

  def bar():
      pass

  def quux(a):
      pass
\end{verbatim}

\newpage
Or, if you want to enforce the abstract nature a bit more,

Example 2:

\begin{verbatim}

class FooClass:

  def bar():
      raise NotImplementedError()

\end{verbatim}

\subsection{Phone numbers}

US phone numbers are 10 digits long. The first three digits are the area code.
The next three are the prefix and the remaining four are the number.  Display them 
with a hyphen between each part, e.g., 206-555-1212.

UK phone numbers are 10 or 11 digits long.  They begin with a three to six
digit area code that begins with a zero.  Next is an optional prefix that is three or four digits long.
The remaining digits are the individual number. Display them with brackets around
the area code, e.g., (020) 3221 8735.

\subsection{Addresses}
US addresses have the following format:

\begin{verbatim}
  recipient
  organisation (optional)
  address line 1
  address line 2 (optional)
  city, state post code
  \end{verbatim}

The state/territory field is a two letter abbreviation.  Postcodes
are nine digits long.

UK addresses have the following format:

\begin{verbatim}

  recipient
  organisation (optional}
  building name (optional)
  address line 
  locality (optional)
  city
  post code
  \end{verbatim}

UK post codes are seven alphanumeric characters, e.g., SO31 4NG.

Note that these address specifications are somewhat simplified.  Real world
addresses are horribly messy.

\end{document}
