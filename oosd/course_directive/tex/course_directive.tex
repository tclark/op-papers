\documentclass{article}
\usepackage{graphicx}
\usepackage{wrapfig}
\usepackage{inconsolata}
\usepackage{enumerate}
\usepackage{hyperref}
\usepackage[margin = 2.25cm]{geometry}



\begin{document}

\begin{figure}
\includegraphics[width=30mm]{../../../resources/images/oplogo.png}
\end{figure}

\title{Course Directive\\IN710 Object Oriented System Development\\Semester One, 2015}
\date{}
\maketitle

\section*{Description}
In this paper, students will develop language-independent skills in object oriented development for medium to large applications. Students will learn current best-practice methods and tools for the design and construction of enterprise systems through a combination of discussion of theoretical principles and extensive coding work.


\section*{Course Information}
\begin{itemize}
  \item 15 Credits
  \item IN610 and IN613
\end{itemize}

\section*{Lecturer}
\begin{tabular}{lr}

  % after \\: \hline or \cline{col1-col2} \cline{col3-col4} ...
  Tom Clark &    \\
     Office: & D311 \\
     Phone: & 470 4356 \\
     Email: & \texttt{tom.clark@op.ac.nz} \\
     GitHub: & \url{https://github.com/tclark} 
\end{tabular}

\section*{Course Dates}
\begin{tabular}{ll}
Term 1 (7 weeks) & 16 February - 2 April\\
Term 2 (9 weeks) & 20 April - 19 June\\
\end{tabular}

\section*{Learning Outcomes}
On completion of this paper you will be able to:
\begin{enumerate}
    \item Understand theoretical and pragmatic issues surrounding design and implementation of large object-oriented software systems.
    \item Analyse a problem statement for a complex software system and design an appropriate class architecture for the problem solution.
    \item Design and implement components of large software systems following appropriate software engineering methodologies and producing industry-quality code.
\end{enumerate}

\section*{Resources}
\begin{itemize}
	\item Course notes, lecture slides, and lab documents are availble in a GitHub repository published at \\ \url{https://github.com/tclark/op-papers}.
	\item You will need a (free) GitHub account to submit manage your work.
	\item If you don't have it installed already, you may want to install Python on your own machine.  Since the lab machines run version 2.7, I recommend you use the same version.  If you really want to use Python 3, let me know.
	\item There is no required text for this paper.  The follwoing books are recommended reading:
		\begin{itemize}
			\item \emph{Design Patterns: Elements of Reusable Object-Oriented Software} by Erich Gamma, Richard Helm, Ralph Johnson, and John Vlissides
			\item \emph{Head First Design Patterns} by Eric Freeman, Elisabeth Robson, Bert Bates, Kathy Sierra
		\end{itemize}
	\item A good Python language reference is also recommended. Do not rely on Stack Overflow.  That way lies madness.
	\item Coding style counts.  Your Python code shall conform to PEP 8 (see
		\url{https://www.python.org/dev/peps/pep-0008/}).
\end{itemize}


\section*{Course Content and Schedule}
This schedule is subject to change based on the needs of the class.

\renewcommand{\arraystretch}{1.5}
\begin{tabular}{|l|c|l|}
\hline
 Week & Week Start & Topic                         \\ \hline
 1    & 16 Feb     & Introduction, Python primer \\ \hline
 2    & 23 Feb     & Core OOAD, SOLID  \\ \hline
 3    &  2 Mar     & Unit Testing   \\ \hline
 4    &  9 Mar     & Strategy Pattern    \\ \hline
 5    & 16 Mar     & Factory Pattern   \\ \hline
 6    & 23 Mar     & Observer Pattern     \\ \hline
 7    & 30 Mar     & Delegates   \\ \hline
 H1   &  6 Apr     & Holiday             \\ \hline
 H2   & 13 Apr     & Holiday             \\ \hline
 8    & 20 Apr     & Anonymous and Lambda   \\ \hline
 9    & 27 Apr     & Multithreading        \\ \hline
 10   &  4 May     & Data Management - XML \\ \hline
 11   & 11 May     & Data Managment - RDBMSs  \\ \hline
 12   & 18 May     & Project Spec and Work \\ \hline
 13   & 25 May     & Project Work   \\ \hline
 14   &  1 Jun     & Project Work   \\ \hline
15   &  8 Jun     & Project Work  (Due Friday)  \\ \hline
 16   & 15 Jun     & Final Preparation and Exam \\ \hline
\end{tabular}

\newpage

\section*{Assessment}
There are three assessments in this paper, weighted as follows:


\begin{tabular}{|l|c|}
\hline
Assessment &  Weighting \\ \hline
 Class Work & 25\% \\ \hline
 Project & 45\% \\ \hline
 Theory Exam & 30\% \\ \hline
\end{tabular}
\subsection*{Submision Requirements}
\begin{itemize}
	\item Detailed assignment requirements, including instructions for submission, will be provided for each assessment.
        \item Weekly checkpoints must be submitted by Tuesday 5:00 PM after the week assigned. They will not be accepted late.
	\item Late projects may be penalised 10\% of the raw mark for each day late (including weekends).
	\item Students should keep a copy of all submitted work.
\end{itemize}

\section*{Criteria for Passing}
You must earn an overall average mark of 50\% or better to pass this paper.

\section*{Course Requirements and Expectations}
\subsection*{Attendance}
\begin{itemize}
 \item Students are expected to attend all classes, both lectures and labs.
 \item If you miss a class you should get notes from another student.
 \item If you cannot attend for two or more consecutive sessions, contact the lecturer.
 \item You must turn up ready for assessments on the due date and at the correct time. No extra time will be scheduled. If you do not turn up, you have failed the assessment.
\end{itemize}

\subsection*{Communication}
Important announcements and discussions about the course, assessments, and scheduling may take place during class sessions.  It is your responsibility to be informed about them.  If you cannot attend a class session, be sure to check with another student.

Your student email is an official communication channel. It is your responsibility to regularly check your student email for important course related material, including changes to class scheduling or assessment details. Not checking will not be accepted as an excuse.

You can manage your email at the Student Hub and download the instructions for forwarding your email at \url{http://www.op.ac.nz/students/student-hub/}

\subsection*{Polytechnic Closure}
In the event that the Polytechnic is closed or has a delayed opening because of snow or bad weather you should not attempt to attend class if it is unsafe to do so. It is possible that your instructor will not be able to attend either, so classes may not physically meet. However, this does not become a holiday. Rather, material will be available on GutHub covering the classes affected by the closure. You are responsible for any material presented in this manner. Information about closure will be posted on the Otago Polytechnic Facebook page \url{https://www.facebook.com/OtagoPoly}.

\subsection*{Group Work and Originality}
Students in the Bachelor of Information Technology degree are expected to hand in original work.  Students are encouraged to discuss
assignments with their fellow students.  However, all assignments are to be completed as individual works unless group work is explicitly involved.
Failure to submit your own unique work will be treated as plagiarism.

\subsection*{Referencing}
Appropriate referencing is required for all work.  Referencing standards will be specified by your instructor.

\subsection*{Plagiarism}
Plagiarism is submitting someone else's work as your own.  Plagiarism offences are taken seriously and an
assessment that has been plagiarised may be awarded a zero mark.  A definition of plagiarism is in the Student Handbook,
available online or at the school office.

\subsection*{Submission Requirements}
All assignments are to be submitted by the time, date, and method given when the assignment is issued.

\subsection*{Extensions}
Extensions are only available for unusual circumstances.  These must be applied for, and approved, prior to the submission deadline.

\subsection*{Impairment}
In case of sickness contact your lecturer or year co-ordinator as soon as possible, preferably before the test or
assignment is due.  The policy regarding the granting of a mark that considers impaired performance requires a medical
certificate and a medical practitioners signature on a form. You may should refer to the guide on impaired performance
on the student handbook.

\subsection*{Appeals}
If you are concerned about any aspect of your assessment, please approach the lecturer in the first instance.  We support
an open door policy and aim to resolve issues promptly.  Further support is available from the Programme
Manager and Head of School. Otago Polytechnic has a formal process for academic appeals if necessary.

\subsection*{Other Documents}
Regulatory documents relating this course can be found on the Polytechnic website.

\subsection*{Special Resources and Requirements}
If you have any special needs, whether they relate to the course material, the exercises, the assessment, or anything in the course -
then \emph{please} let your instructor know as soon as possible.

\end{document}
