% Beamer slide template prepared by Tom Clark <tom.clark@op.ac.nz>
% Otago Polytechnic
% Dec 2012

\documentclass[10pt]{beamer}
\usetheme{Dunedin}
\usepackage{graphicx}
\usepackage{fancyvrb}
\newcommand\codeHighlight[1]{\textcolor[rgb]{1,0,0}{\textbf{#1}}}

\title{Introduction to Functional Progamming}
\author[IN710]{OOSD}
\institute[Otago Polytechnic]{
  School of Information Technology \\
  Otago Polytechnic \\
  Dunedin, New Zealand \\
}
\date{}
\begin{document}

%----------- titlepage ----------------------------------------------%
\begin{frame}[plain]
  \titlepage
\end{frame}

%----------- slide --------------------------------------------------%
\begin{frame}
  \frametitle{What is object-oriented programming?}

  Typically prgramming can be thought of as managing \emph{state}.
\begin{itemize}
  \item We have some variables that hold values.
  \item Those values change over time as the program is executed.
  \item Our programs are sets of instructions on how to change state.
  \item The value of those variables at any point in time characterises the
	  program's state.
\end{itemize}

\end{frame}
%----------- slide --------------------------------------------------%
\begin{frame}
  \frametitle{What is object-oriented programming?}

  The difficulty with this type of programming in that it is often 
  difficult to reason about a program's state.
\begin{itemize}
  \item A bug in a program occurs when a variable doesn't hold the
	  corect value.  The program's state is incorrect.
  \item In OO programming, we use \emph{objects} to encapsulate state.
  \item Objects protect their states and provide methods to control changes 
	  to their states.  
\end{itemize}

\end{frame}
%----------- slide --------------------------------------------------%
\begin{frame}
  \frametitle{What is functional programming?}

  The aim of functional programming is to allow us to program without state.
  Instead of supplying instructions about how to change state, our programs
  describe to desired solution in terms of the application of functions.

\end{frame}
%----------- slide --------------------------------------------------%
\begin{frame}
  \frametitle{Functional programming principles}

   1.  Immutable data:  Once we assign a value to a variable we avoid 
   changing that value, since that gets us back to managing state.
\end{frame}
%----------- slide --------------------------------------------------%
\begin{frame}
  \frametitle{Functional programming principles}

   2. Pure functions:  Pure just functions take arguments and return a values. 
   They are \emph{deterministic}, meaning they always return the same values when 
   given the same arguments.  Pure functions avoid \emph{side effects}, since
   side effects generally involve state.
\end{frame}
%----------- slide --------------------------------------------------%
\begin{frame}
  \frametitle{Functional programming principles}

   3.  First class functions: Functions are just values that can be passed around 
   just like we pass around an integer value in a program.
\end{frame}
%----------- slide --------------------------------------------------%
\begin{frame}
  \frametitle{Functional programming principles}

   4.  Higher order functions: We write fuctions that operate on other 
   functions.
\end{frame}
%----------- slide --------------------------------------------------%
\begin{frame}[fragile]
  \frametitle{Example}

 
\begin{Verbatim}[commandchars=\\\[\]]
    def doubler(f):
        def tmp(x):
            return 2 * f(x)
        return tmp
\end{Verbatim}

\end{frame}
%----------- slide --------------------------------------------------%
\begin{frame}
  \frametitle{Why bother with FP}

  \begin{itemize}
	  \item Functional programming facilitates very rapid development
		  since fuctional code is typiclaly very expressive.
	  \item Functional programs have fewer bugs and are easier to debug.
  \end{itemize}
\end{frame}


%----------- slide --------------------------------------------------%
\begin{frame}
  \frametitle{Functional programming languages}

If our goal is to do functional programming, then there are languages 
that implement these principles and facilitate programming functionally.
\begin{itemize}
	\item Lisp
	\item Haskell
	\item Clojure
	\item Scheme
	\item Erlang
	\item and many others
\end{itemize}
\end{frame}
%----------- slide --------------------------------------------------%
\begin{frame}
  \frametitle{Functional programming in Python}

  We can also write functional programs in Python.  The language
  provides features that facilitate this.  We can write purely functional
  programs, but more commonly we use a mix of OO and functional constructs
  as appropriate.
  \vspace{5mm}

  If we simple realise that managing state is difficult and we learn to do it 
  in a careful and deliberate way, we get some of the benefits of FP without
  having to discard OO.
  \end{frame}

%----------- slide --------------------------------------------------%
\begin{frame}[fragile]
  \frametitle{Functional tools: Lambda}

  In FP we often write short, and frequently anonymous functions. 
  \emph{Lambda expressions} are useful for this.

  The following are equivalent.
 
\begin{Verbatim}[commandchars=\\\[\]]
    addone = lambda x: x + 1

    def addone(x):
        return x + 1

\end{Verbatim}

\end{frame}

%----------- slide --------------------------------------------------%
\begin{frame}
  \frametitle{Functional tools: List operations}

  When we program without state, we can't have loops anymore.  Instead,
  we operate (recursively) on lists, often with higher-order functions.
  Three important one are

  \begin{itemize}
	  \item map
	  \item filter
	  \item reduce
  \end{itemize}
  \end{frame}

%----------- slide --------------------------------------------------%
\begin{frame}[fragile]
  \frametitle{Functional tools: map}

  \texttt{map} takes a function and a list and returns a new list
  produced by applying the function to each element in the original
  list.  
\begin{Verbatim}[commandchars=\\\%\%]
  map(f, [1, 2, 3]) -> [f(1), f(2), f(3)]

  map(lambda x: 2 * x, [1, 2, 3]) # returns [2, 4, 6]
  \end{Verbatim}
\end{frame}
%----------- slide --------------------------------------------------%
\begin{frame}[fragile]
  \frametitle{Functional tools: filter}

  \texttt{filter} takes a boolean function and a sequence and returns a new list
  that contains the elements of the original sequence whose members satisfy the 
  boolean condition.
  \begin{Verbatim}[commandchars=\\\!\!]

 filter(lambda x: x % 2 == 0, range(10)) #returns [2, 4, 6, 8]
  \end{Verbatim}
\end{frame}
%----------- slide --------------------------------------------------%
\begin{frame}[fragile]
  \frametitle{Functional tools: reduce}

  \texttt{reduce} is used to reduce a list of values down to a single value. It
  takes a function of two values and a sequence and returns the result obtained by successively
  applying the function to items in the sequence and the previously computed value.
  \begin{Verbatim}[commandchars=\\\!\!]

  reduce(lambda x, y : x + y, range(10)) #returns 45
  \end{Verbatim}
\end{frame}
%----------- slide --------------------------------------------------%
\begin{frame}[fragile]
  \frametitle{Functional tools: reduce}

  \texttt{reduce} has an optional 3rd argument that is an initial value to
  use in the reduction function.
  \begin{Verbatim}[commandchars=\\\!\!]

  reduce(lambda x, y : x + y, range(10)) #returns 45
  reduce(lambda x, y : x + y, range(10), 5) #returns 50
  \end{Verbatim}
\end{frame}

\begin{frame}
	\frametitle{Exercises}
	\begin{enumerate}
		\item Write a lambda expression that computes factorials.
			(Use \texttt{reduce}.)
		\item Write a function that takes a string and a character
			and returns a new string that is the original string with
			all occurences of the character removed.
		\item Write a function that is similar to the one above, but 
			that returns the number of occurences of the character.
		\item Write a function that takes a string and a character
			and returns the number of words that start with the
			character.
		\item Write a function that takes a string and a character
			and returns a new string that is the original string with
			all occurences of the character in uppercase.
		\item Write your own version of \texttt{map} (call it
			\texttt{mymap}) using a loop.
		\item Rewrite \texttt{mymap} using recursion.
	\end{enumerate}
\end{frame}
\end{document}
