\documentclass{article}
\usepackage{graphicx}
\usepackage{wrapfig}
\usepackage{inconsolata}
\usepackage{enumerate}
\usepackage{hyperref}
\usepackage[margin = 2.25cm]{geometry}



\begin{document}

\begin{figure}
\includegraphics[width=30mm]{../../../resources/images/oplogo.png}
\end{figure}

\title{Unit Testing Exercises\\IN710 Object Oriented System Development}
\date{}
\maketitle

\section*{Description}
These exercises will give you some practice applying unit testing and TDD principles
in Python. Push your code to your GitHub repository by the first class of next week.
Be prepared to present your code in class.

Get started on this exercise by pulling down the \texttt{unit\_testing\_exercises} subdirectory
in the \texttt{week03} directory on the lecturer's GitHub.

\section{Person class}
The exercises include a stub of a \texttt{Person} class along with a complete 
\texttt{TestPerson} unittest test case.  The tests in \texttt{TestPerson} 
effectively define what the \texttt{Person} class should do.  Based on the 
tests, code up a complete \texttt{Person} class so that all the tests pass.

\section{MySet class}
Suppose that there was no built-in set type in Python and we wanted
to implement one.  Do this by implementing the methods in the 
\texttt{MySet} class.  Implement a full set of tests in the \texttt{TestMySet}
class.

Tip:  Start by implementing the constructor so that it handle and empty list
as an argument.  Later, go back and handle the case where the list passed to
the constructor is nonempty.

\end{document}
